\documentclass[cal1spr16Lectures.tex]{subfiles}

\begin{document}

%\section[Week 11]{Week 11: 4-8 Apr}

% % % 
\subsubsection{\bf Wednesday 6 April}
% % %

\begin{frame}[allowframebreaks]{Wed 6 Apr}
\begin{itemize}%\footnotesize
\item Exam 3: Friday.  Covers \S 3.10-4.6.  You will need a scientific calculator. 
\end{itemize}
\end{frame}

% % %
\subsection{Exam \#3 Review}
% % %

% % %
\begin{frame}[allowframebreaks]{Exam \# 3 Review}\small
\begin{itemize}
\item \S 3.10 Derivatives of Inverse Trig Functions
\begin{itemize}\footnotesize
	\item Know the derivatives of the six inverse trig functions.
	\item Also: You are responsible for every derivative rule and every derivative formula we have covered this semester.
\end{itemize} 
%\end{frame}
%
% % %
%\begin{frame}
%\frametitle{\small 
\item \S 3.11 Related Rates
\begin{itemize}\footnotesize
	\item Know the steps to solving related rates problems, and be able to use them to solve problems given variables and rates of change.
	\item Be able to solve related rates problems.  If, while doing the HW (paper or computer), you were provided a formula in order to solve the problem, then I will do the same.  If you were not provided a formula while doing the HW (paper or computer), then I also will not provide the formula.
\end{itemize}
%\end{frame}
%\end{frame}
%\framebreak
\begin{exe} An inverted conical water tank with a height of 12 ft and a radius of 6 ft is drained through a hole in the vertex at a rate of 2 ft$^{\text{3}}$/sec.  What is the rate of change of the water depth when the water depth is 3 ft? \end{exe}
% % %
%\begin{frame}
%\frametitle{\small 
%\framebreak
\item \S 4.1 Maxima and Minima
%\small
	\begin{itemize}\footnotesize
	\item Know the definitions of maxima, minima, and what makes these points local or absolute extrema (both analytically and graphically).
	\item Know how to find critical points for a function.
	\item Given a function on a given interval, be able to find local and/or absolute extrema.
	\item Given specified properties of a function, be able to sketch a graph of that function.
	\end{itemize}
%\end{frame}
%\framebreak
% % %
%\begin{frame}
%\frametitle{\small 
\item \S 4.2 What Derivatives Tell Us
%\footnotesize
	\begin{itemize}\footnotesize
	\item Be able to use the first derivative to determine where a function is increasing or decreasing.
	\item Be able to use the \alert{First Derivative Test to identify local maxima and minima}.  Be able to explain in words how you arrived at your conclusion.
	\item Be able to find critical points, absolute extrema, and inflection points for a function.
	\item Be able to use the second derivative to determine the concavity of a function.
	\item Be able to use the \alert{Second Derivative Test to determine whether a given point is a local max or min}.  Be able to explain in words how you arrived at your conclusion.
	\item Know your Derivative Properties!!! (see Figure 4.36 on p.\ 256)
	\end{itemize}
%\end{frame}
%\framebreak
% % %
%\begin{frame}
%\frametitle{\small 
\item \S 4.3 Graphing Functions
%\small
	\begin{itemize}\footnotesize
	\item Be able to find specific characteristics of a function that are spelled out in the Graphing Guidelines on p.\ 261 (e.g., know how to find $x$- and $y$-intercepts, vertical/horizontal asymptotes, critical points, inflection points, intervals of concavity and increasing/decreasing, etc.).  
	\item Be able to use these specific characteristics of a function to sketch a graph of the function.
	%\item TIP: If you are looking for intervals for increasing/decreasing or concave up/down, you should {\it treat} the naughty points as critical points and/or possible points of inflection.
	\end{itemize} 
%\end{frame}
%\framebreak
% % %
%\begin{frame}
%\frametitle{\small 
\item \S 4.4 Optimization Problems
%\small
	\begin{itemize}\footnotesize
	\item Be able to solve optimization problems that maximize or minimize a given quantity.
	\item Be able to identify and express the constraints and objective function in an optimization problem.
	\item Be able to determine your interval of interest in an optimization problem (e.g., what range of $x$-values are you searching for your extreme points?)
	\item {\bf As to formulas, the same comment made above with respect to formulas for related rates problems applies here as well.}
	\end{itemize}
%\end{frame}
%\framebreak
\begin{exe} What two nonnegative real numbers $a$ and $b$ whose sum is 23 will 
		\begin{itemize}
		\item[(a)] minimize $a^2+b^2$?
		\item[(b)] maximize $a^2+b^2$?
		\end{itemize}	
\end{exe}
% % %
%\begin{frame}
%\frametitle{\small
%\framebreak 
\item \S 4.5 Linear Approximation and Differentials
%\small
	\begin{itemize}\footnotesize
	\item Be able to find a linear approximation for a given function.
	\item Be able to use a linear approximation to estimate the value of a function at a given point.
	\item Be able to use differentials to express how the change in $x$ ($dx$) impacts the change in $y$ ($dy$).
	\end{itemize}
%\end{frame}
%\framebreak
%\frametitle{4.6 Mean Value Theorem}
%\small
%\begin{itemize}
\item \S 4.6 Mean Value Theorem (for Derivatives)
	\begin{itemize}\footnotesize
	\item Know and be able to state Rolle's Thm and the Mean Value Thm, including knowing the hypotheses and conclusions for both.
	\item Be able to apply Rolle's Thm to find a point in a given interval.
	\item Be able to apply the MVT to find a point in a given interval.
	\item Be able to use the MVT to find equations of secant and tangent lines.
	\end{itemize}
%\framebreak
\begin{exe}[s]
Determine whether the Mean Value Theorem (or Rolle's Theorem) applies to the following functions.  If it does, then find the point(s) guaranteed by the theorem to exist.
	\begin{itemize}\footnotesize
	\item[(1)] $f(x)=\sin{(2x)}$ on $\left[0,\textstyle\frac{\pi}{2}\right]$
	\item[(2)] $g(x)=\ln{(2x)}$ on $\left[1,e\right]$
	\item[(3)] $h(x)=1-\left|x\right|$ on $\left[-1,1\right]$
	\end{itemize}
\end{exe}
\framebreak
\begin{exe}[s] 
	\begin{itemize}\footnotesize
	\item[(4)] $j(x)=x+\textstyle\frac{1}{x}$ on $\left[1,3\right]$
	\item[(5)] $k(x)=\textstyle\frac{x}{x+2}$ on $\left[-1,2\right]$
	\end{itemize}
\end{exe}	
%\end{frame}
\end{itemize}
\end{frame}

\end{document}