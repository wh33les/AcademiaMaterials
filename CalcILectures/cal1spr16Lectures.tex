\documentclass[12pt]{beamer}
\usetheme{Warsaw}
\usecolortheme{beaver,lily}
\usefonttheme{professionalfonts}
\setbeamertemplate{frametitle continuation}[from second][(cont.)]
% \setbeamersize{text margin left=50pt,text margin right=50pt} 
 
\setbeamerfont{section in toc}{size=\fontsize{10}{11}\selectfont}
\setbeamerfont{subsection in toc}{size=\fontsize{8}{9}\selectfont}
\setbeamerfont{subsubsection in toc}{size=\fontsize{6}{7}\selectfont}
\setbeamercolor{section number projected}{bg=red!50!black,fg=white}
\setbeamercolor{subsection in toc}{fg=red!50!black}
\setbeamercolor{subsubsection in toc}{fg=black}
\setbeamercolor{titlelike}{fg=red!50!black}
\setbeamertemplate{subsection in toc}{
	\vspace{0.25pc}
	\hspace{-20pt}\inserttocsubsection
	
	} % don't get rid of the above line of space!
\setbeamertemplate{subsubsection in toc}{
	%\hspace{20pt}
	\textbullet\ \inserttocsubsubsection
	
	} % don't get rid of the above line of space!

%\usepackage{amsmath}	
\usepackage{subfiles}
\usepackage{multicol}
\usepackage{url}
\usepackage{tikz}
\usepackage{tkz-euclide}
\usetkzobj{all}
%\usepackage{xcolor}
\usepackage{comment}
\usepackage{hyperref}
%\usepackage{changepage}
	
%\begin{comment}
\AtBeginPart{
	\begin{frame}[allowframebreaks]{Part \thepart. \insertpart}
	%\begin{multicols}{2}
	\tableofcontents[sectionstyle=show/show]
	%\end{multicols}
	\end{frame}
	}
\AtBeginSubsection{
	\begin{frame}[allowframebreaks]%{\insertsubsection}
 	\begin{multicols}{2}
	\tableofcontents[currentsection, subsectionstyle=show/shaded/hide]
	\end{multicols}
	\end{frame}
	}
%\end{comment}	

\theoremstyle{plain}
\newtheorem{thm}{Theorem}

\theoremstyle{definition}
\newtheorem{que}{Question}
\newtheorem{ex}{Example}
\newtheorem{exe}{Exercise}
\newtheorem{dfn}{Definition}

%\everymath{\displaystyle}

% % % % % % % % % % % % % % % % % % % %

\title[Cal I Spring 2016]{Calculus I (Math 2554)}
\subtitle{Spring 2016}
\author[Wheeler]{\footnotesize Dr. Ashley K. Wheeler}
\institute{University of Arkansas}
\date{\footnotesize{\it last updated:} \today}
\logo{The base for these slides was done by Dr. Shannon Dingman, later encoded into \LaTeX\ by Dr. Brad Lutes and modified/formatted by Dr. Ashley K. Wheeler.}

% % % % % % % % % % % % % % % % % % % %
\begin{document}
% % %
\frame{\titlepage}
% % %

%\begin{comment}
% % %
\begin{frame}[allowframebreaks]{Table of Contents}
%\begin{adjustheight}{-1.5em}{-1.5em}
%\begin{multicols}{2}
%\begin{minipage}%{\dimexpr\textwidth+2\relax}
 %  \raggedright3em
\vspace{1pc} 
\tableofcontents[currentsection]
\vspace{-20pc}
{
	\setbeamertemplate{subsubsection in toc}{} %%% This is how I suppress the subsubsections 

	%%% To include Parts in the TOC I use the itemize environment.
	\begin{itemize}\color{red!50!black}\bf
		\item[]\hspace{-30pt}Part I: Limits
		\vspace{0.25pc}
		%\tableofcontents[part=1]
		\item[]\hspace{-30pt}Part II: Derivatives
		%\tableofcontents[part=2]
		\item[]\hspace{-30pt}Part III: Applications and Story Problems
		%\tableofcontents[part=3]
		\item[]\hspace{-30pt}Part IV: Introduction to Integrals
		%\tableofcontents[part=4]	
	\end{itemize}
} %%% Now back to the global settings for TOC.

\end{frame}
%\end{comment}

% % % 
\subsubsection{Tips for Success}
% % %

% % %
\begin{frame}{Tips for Success}\footnotesize
\begin{itemize}
\item Attend class every day.  \alert{Participate} in math discussions.  Do the lecture-cises fully, not just on a scratch paper.  
\item Don't get behind on MLP homeworks.  Stay on top of the book problems.
\item Find a study partner(s) to meet with on a regular basis.  Don't be afraid to seek further assistance (tutoring, office hours, etc.) if you are struggling.
\item high school calculus $\neq$ college calculus
\item REMEMBER... THE TERM STARTS TODAY!  SO DOES THE EVENTUAL EARNING OF YOUR FINAL GRADE!!!
\end{itemize}
\end{frame}

%%% Days pasted together

% % % % % % % % % %
\part{Limits}
% % % % % % % % % %
\section[Week 1]{19-22 January}
% % % % %

\subfile{week1Wed}
\subfile{week1Fri}

% % % % %
\section[Week 2]{25-29 January}
% % % % %

\subfile{week2Mon}
\subfile{week2Wed}
\subfile{week2Fri}

% % % % %
\section[Week 3]{1-5 February}
% % % % %

\subfile{week3Mon}
\subfile{week3Wed}
\subfile{week3Fri}

% % % % %
\section[Week 4]{8-12 February}
% % % % %

\subfile{week4Mon}
\subfile{week4Wed}

% % % % % % % % % %
\part{Derivatives}
% % % % % % % % % %

% % % % %
\section[Week 5]{15-19 February}
% % % % %

\subfile{week5Mon}
\subfile{week5Wed}

% % % % %
\section[Week 6]{22-26 February}
% % % % %

\subfile{week6Mon}
\subfile{week6Wed}

% % % % %
\section[Week 7]{29 Feb -- 4 March}
% % % % %

\subfile{week7Wed}

% % % % % % % % % %
\part{Applications and Story Problems}
% % % % % % % % % %
\section[Week 8]{7-11 March}
% % % % %

\subfile{week8Mon}
\subfile{week8Fri}

% % % % %
\section[Week 9]{14-18 March}
% % % % %

\subfile{week9Mon}
\subfile{week9Wed}
\subfile{week9Fri}

% % % % %
\section[Week 10]{28 Mar -- 1 April}
% % % % %

\subfile{week10Mon}
\subfile{week10Wed}
\subfile{week10Fri}

% % % % %
\section[Week 11]{4-8 April}
% % % % %

\subfile{week11Mon}
\subfile{week11Wed}

% % % % % % % % % %
\part{Introduction to Integrals}
% % % % % % % % % %

% % % % %
\section[Week 12]{11-15 April}
% % % % %

\subfile{week12Mon}
\subfile{week12Wed}
\subfile{week12Fri}

% % % % %
\section[Week 13]{18-22 April}
% % % % %

\subfile{week13Mon}
\subfile{week13Wed}
\subfile{week13Fri}

% % % % %
\section[Week 14]{25-29 April}
% % % % %

\subfile{week14Mon}
\subfile{week14Wed}

% % % % %
\section[Week 15]{2-4 May}
% % % % %

\subfile{week15Mon}
\subfile{week15Wed}

% % % % % % % % % % % % % % % % % % % %

%%% Weeks pasted together
%\begin{comment}
%\subfile{week1}
%\subfile{week2}
%\subfile{week3}
%\subfile{week4}
%\subfile{week5}
%\subfile{week6}
%\subfile{week7}
%\end{comment}

% % % % % % % % % % % % % % % % % % % %
\end{document}