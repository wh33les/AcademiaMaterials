\documentclass[cal1spr16Lectures.tex]{subfiles}

\begin{document}

%\section[]{}

% % %
\subsection[Final Preparation]{Final Preparation}
% % %

% % %
\begin{frame}[allowframebreaks]{Final Preparation}\footnotesize
Perparation for final: Be sure to download the study guide for the final and note the sections to focus on (e.g., ignore 4.3, 5.1, 5.2).  Be prepared to do:
\begin{itemize}\footnotesize
\item Integration (power rule, substitution) -- you'll have time to check these using differentiation!
\item Related Rates
\item Optimization
\item Use of First and Second Derivative Test
\item Derivatives of trig functions, inverse trig functions, log and exponential functions
\item Use of derivative to find equations of tangent lines
\item Limits (using analytical methods and L'H\^opital's)
\end{itemize}
\end{frame}

% % %
\begin{frame}
Preparation for final:
\begin{itemize}\small
\item In general, anything that is on the study guide is fair game!!!
\item WATCH YOUR NOTATION!!!! (e.g., limit notations, derivative notation, integral notation, etc.)
\item WATCH YOUR DIRECTIONS!!!!! (e.g., finding limits analytically)
\item CHECK YOUR WORK!!!!! (You should have time!!)
\end{itemize}
\end{frame}

% % %
\begin{frame}\small
A good place to start is reworking problmes from the 5 exams (4 hourly tests plus midterm).  This gives you a wide (yet still incomplete) scope of the problems we have done.

\vspace{0.5pc}
Other things you can do to prepare for the final:
\begin{itemize}\footnotesize
\item Examine the Study Plan on Mylabsplus to see areas where you struggled on Computer HWs
\item Review Completed Paper HWs (or finish paper HWs!)
\item Go back over problems worked in class, on quizzes, and on drill exercises
\end{itemize}
\end{frame}

% % %
\subsubsection[About the Test]{About the Test}
% % %

% % %
\begin{frame}{\small About the Test}
\begin{itemize}
\item It is cumulative!!! However, the course has built to this point, so expect more from material since the midterm than before.
\item 20 questions in 2 hours
\item Grades should be completed by the end of the week (Friday, 13 May PM)
\end{itemize}
\end{frame}

% % %
\subsubsection[Advice for the FINAL]{Advice for the FINAL}
% % % 

% % %
\begin{frame}{\small Advice for the FINAL}
\begin{itemize}\footnotesize
\item $+C$s, $dx$s, $\lim$, units, etc. should be included in your answers {\it or else}.  Don't try to round answers unless it is for a story problem, in which case, you should say ``approximately".
\item ``Definition of Derivative" = the definition with limits
\item Practice limits and l'H\^{o}pital's Rule so you know which is the quickest technique.
\item ``Mean Value Theorem for Derivatives" = MVT from \S 4.6.
\item $\arctan=\tan^{-1}$, etc.
\item Use the Continuity Checklist for questions about continuity.
\item Use limits for questions about vertical asyptotes and end behavior.
\end{itemize}
\end{frame}

% % %
\subsubsection{Easter Egg-xercises}
% % %

% % %
\begin{frame}{\small Easter Egg-xercises}
\begin{exe}[s] \begin{itemize}
\item[1.] Find the 101st derivative of $y=\cos{7x}$ at $x=0$.
\item[2.] For what values of the constants $a$ and $b$ is $(-1,2)$ a point of inflection on the curve $y=ax^3+bx^2-8x+2$?
\end{itemize}\end{exe}
\end{frame}

\begin{comment}
\end{comment}

\end{document}