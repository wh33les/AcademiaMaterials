%\documentclass[notes=only]{beamer}   % only notes
\documentclass[%handout,
	%notes, 
	12pt]{beamer}
%%% The following snipet is from https://tex.stackexchange.com/questions/6348/problem-with-beamers-pause-in-alignments/6352
\mode<presentation> { \setbeamercovered{%transparent
	invisible} }
\setbeamertemplate{navigation symbols}{}
\makeatletter
\def\beamerorig@set@color{%
  \pdfliteral{\current@color}%
  \aftergroup\reset@color
}
\def\beamerorig@reset@color{\pdfliteral{\current@color}}
\makeatother
%%% ... attempt to make pause work in an align environment because I really wanted this effect
\usepackage{lmodern} % this package avoids a pdftex font compile error

\usepackage{xcolor}
%\usepackage{subfiles}
%\usepackage{multicol}
\usepackage{url}
\usepackage{tikz}
%\usepackage{tkz-euclide}
%\usetkzobj{all}
\usepackage{comment}
\usepackage[mathscr]{euscript}
%\usepackage{enumerate}
%\usepackage{paralist}
\usepackage[all,pdf]{xy}
	\CompileMatrices
\usepackage{cancel}	
\usepackage{url}
	
\usetheme{Warsaw}
\colorlet{powCol}{cyan!70!blue}
\colorlet{bkCol}{cyan}
\usecolortheme[named=powCol]{structure}
\usecolortheme{beaver}
\setbeamercolor{titlelike}{fg=powCol}	
\setbeamercolor{palette primary}{fg=powCol}
\setbeamercolor{palette secondary}{fg=powCol}
\setbeamercolor{palette tertiary}{fg=powCol}
\setbeamercolor{palette quaternary}{fg=powCol}

% % %		
\theoremstyle{plain}
\newtheorem{thm}{Theorem}
\newtheorem{prop}{Proposition}
\newtheorem{lem}{Lemma}
\newtheorem{conj}{Conjecture}
\newtheorem{cor}{Corollary}
\theoremstyle{definition}
\newtheorem{que}{Big question}
\newtheorem{ex}{Example}
\newtheorem{exe}{Exercise}
\newtheorem{dfn}{Definition}

%\everymath{\displaystyle}

\newcommand{\N}{\mathbb N}
\newcommand{\Z}{\mathbb Z}
\newcommand{\C}{\mathbb C}
\newcommand{\Q}{\mathbb Q}
\newcommand{\vect}[1]{\mathbf{#1}}
\DeclareMathOperator{\pr}{\mathfrak B}
\DeclareMathOperator{\Var}{\mathcal V}
\DeclareMathOperator{\I}{I}
\DeclareMathOperator{\hgt}{ht}
\DeclareMathOperator{\rank}{rank}
\DeclareMathOperator{\Grass}{Grass}
\DeclareMathOperator{\Graph}{Graph}
\DeclareMathOperator{\codim}{codim}
\DeclareMathOperator{\col}{col}
\DeclareMathOperator{\fr}{fr}
\DeclareMathOperator{\SL}{SL}
\DeclareMathOperator{\cl}{c\ell}
\DeclareMathOperator{\GL}{GL}
\DeclareMathOperator{\ch}{char}
\DeclareMathOperator{\row}{row}
\DeclareMathOperator{\Pf}{Pf}
\DeclareMathOperator{\Prj}{\mathbb P}
\DeclareMathOperator{\inj}{\hookrightarrow}
\DeclareMathOperator{\Mat}{Mat}
\DeclareMathOperator{\SP}{Sp}
\DeclareMathOperator{\sign}{sign}
\DeclareMathOperator{\rref}{rref}

% % % % % % % % % % 

\title[Defining equations for matroid varieties]{Defining equations for matroid varieties -- using the Grassmann-Cayley algebra}
\subtitle{Virtual Inspiring Talks Series%\small  \\ 
	%\footnotesize 
	}% \\
	% \\
	%
\author[A. K. Wheeler]{Ashley K. Wheeler%
	%\vspace{-0.75pc}
	}
\institute{\color{powCol}Mount Holyoke College%
	%\vspace{-0.6pc}
	}
\date{%\small 
	Fall 2020}
\logo{background from \emph{Euclidea}}

% % % % % 
\begin{document}

\usebackgroundtemplate{
	\includegraphics[width=\paperwidth,height=\paperheight]{JTS2020PascalBkgd.eps}
	}
	
% % %
\begin{frame}
%\large
\titlepage
\end{frame}

% % % 
\begin{frame}
\begin{flushright} \textcolor{powCol}{Thank-you for the invitation to speak!} \end{flushright}

\vspace{0.075pc}
\hrule

\pause
\vspace{1pc}
\textcolor{powCol}{About me:} \pause
\vspace{0.25pc}
\begin{itemize}
\item Biracial, 2nd generation college student \pause
\item Military family \pause
\item Education: USAFA \pause $\to$ KState \pause $\to$ Michigan
\end{itemize}
\end{frame}

% % %
\begin{frame}{}{}
\textcolor{powCol}{About the project:} \pause
\vspace{0.25pc} 
\begin{itemize}
\item Joint w/ Jessica Sidman (MHC) and Will Traves (USNA). 
\[
\includegraphics[scale=0.025]{sidman}
\quad
\includegraphics[scale=0.25]{traves}
\]

\pause
\vspace{-0.5pc}
\item Paper accepted to \textit{Journal of Combinatorial Theory, Series A} this year (and we've already got a citation!). \pause
\item \textcolor{powCol}{Goal:} To find defining equations for matroid varieties.
\end{itemize}
\end{frame}
	
% % % 
\begin{frame}{}{}
\textcolor{powCol}{Outline:}
\vspace{0.25pc}
\begin{itemize}
\item Front matter
\item Introduction: What is a matroid?
\item Problem: Defining equations for matroid varieties
\item Paradigm shift: Matroids as point configurations
\item Tool: Grassmann-Cayley algebra
\item Back matter
\end{itemize}

\pause
\vspace{1pc}
\textcolor{powCol}{Advice:} Refer to Ravi Vakil's \textit{3 Things} when viewing this (or any other) math talk!
\end{frame}
	
% % % % %
\section{Introduction: What is a matroid?}

% % % 
\begin{frame}{Introduction}{What is a matroid?}
{\color{powCol}Q:} Which collections of 3 columns form a basis for the column space of $A$?  
\[
A=
\begin{pmatrix}
-1 & 1 & 2 & 3 & 4 \\
-1 & 1 & 2 & 0 & -2 \\
1 & 1 & 1 & 1 & 1
\end{pmatrix}
\]
\pause How can we tell?
\end{frame}

% % % 
\begin{frame}
Could row reduce A first.  
\[
\begin{pmatrix}
-1 & 1 & 2 & 3 & 4 \\
-1 & 1 & 2 & 0 & -2 \\
1 & 1 & 1 & 1 & 1
\end{pmatrix}
\to
\rref A = 
\begin{pmatrix}
1 & 0 & -\frac{1}{2} & 0 & \frac{1}{2} \\
0 & 1 & \frac{3}{2} & 0 & -\frac{3}{2} \\
0 & 0 & 0 & 1 & 2
\end{pmatrix}
\]

\pause 
\vspace{0.5pc}
Preferred way: Look for non-zero $\mathbf 3$\textbf{-minors} of $A$ (or of $\rref A$).  

\vspace{1pc}
{\color{powCol}Fact:} Columns of an $r\times r$ matrix are linearly independent if and only if the determinant of the matrix is non-zero.
\end{frame}

% % %
\begin{frame}
Number the columns of $A$ 1 through 5.  Here are the 3-minors, according to their column indices:
\begin{align*}
\{1,2,3\}\to 0 & \qquad \{1,2,4\}\to -6 & \{1,2,5\}\to -12 \\ 
\{1,3,4\}\to -9 & \qquad \{1,3,5\}\to -18 & \{1,4,5\}\to -9\\
\{2,3,4\}\to 3 & \qquad \{2,3,5\}\to -6 & \{2,4,5\}\to -11 \\
\{3,4,5\}\to 0 & & 
\end{align*}

\pause
\vspace{0.5pc}
The \textbf{matroid} $\mathscr M_A$ on $A$ is given by the set: 

\vspace{-1.75pc}
\[
\mathscr B = \{\text{all $3$-tuples except $\{1,2,3\}$ and $\{3,4,5\}$}\}\subset \{1,\dots,5\}
\]

\vspace{-0.5pc}
The sets in $\mathscr B$ are called \textbf{bases} for the matroid $\mathscr M_A$.
\end{frame}	

% % % 
\begin{frame}{}{}
{\color{powCol}
Matroids are a generalization of the collections of linearly independent columns of a matrix.
}

\pause
\vspace{1pc}
{\color{powCol}Applications:} Where are matroids used? \pause
\begin{itemize}
\item \emph{Combinatorial optimization:} artificial intelligence, machine learning, software engineering; example: travelling salesman problem \pause
\item \emph{Coding theory:} error correcting, data compression, cryptography \pause
\item \emph{Network theory:} particle physics, biology, social networks; example: bridges of K\"onigsberg problem
\end{itemize}
\end{frame}

% % % 
%\begin{frame}
%\vspace{-0.5pc}
%\begin{center}
%\includegraphics[scale=0.625]{Konigsberg_bridges}
%\end{center}
%\end{frame}

% % % % %
\section{Problem: Defining equations for matroid varieties}
	
% % % 
\begin{frame}{Problem}{Defining equations for matroid varieties}
{\color{powCol}Q:} Given a matroid on a matrix $A$, what other matrices have the same matroid as $A$? (e.g., $\rref A$)  

\pause
\vspace{1pc}
Consider this \textbf{generic matrix}:
\[
X=\begin{pmatrix}
x_{11} & x_{12} & x_{13} & x_{14} & x_{15} \\
x_{21} & x_{22} & x_{23} & x_{24} & x_{25} \\
x_{31} & x_{32} & x_{33} & x_{34} & x_{35}
\end{pmatrix}
\]
We want all possible $x$-values that will give the same matroid (i.e. same columns as bases) as $A$.  
\end{frame}

% % % 
\begin{frame}{}{}
\textbf{Defining equations} are a system of equations in the $x$'s whose solution set minimally cuts out all of the matrices with matroid $\mathscr M_A$.  

\pause
\vspace{1pc}
Call this solution set $\Var_A$, the \textbf{matroid variety} on $A$.
\end{frame}

% % %
\begin{frame}{}{}
\[
\includegraphics[scale=0.1]{mnev}
\
\includegraphics[scale=0.2]{sturmfels}
\
\includegraphics[scale=0.1]{knutson}
\
\includegraphics[scale=0.2]{lam}
\
\includegraphics[scale=0.2]{speyer}
\]

{\color{powCol} Mn\"ev (1985), Sturmfels (1989), Knutson, Lam, \& Speyer (2013): \textbf{All ``hell'' breaks loose when we try to find defining equations for $\Var_A$.}} 

\pause
\vspace{1pc}
Why?  \pause Two reasons: \pause
\begin{itemize}
\item The \emph{Zariski topology} (beyond the scope of this talk). \pause
\item The equations are not obvious (as we shall see).
\end{itemize} 
\end{frame}

% % % % %
\section{Paradigm shift: Matroids as point configurations}

% % %
\begin{frame}{Paradigm shift}{Matroids as point configurations}
{\color{powCol}Idea:} Think of the columns of $A$ as points in the plane. 

\pause
\vspace{1pc}
The columns of the matrix $A$ are vectors in 3-space, that span lines through the origin.

\vspace{1pc}
If we cut these lines with a plane then the lines become points in the plane.

\pause
\vspace{1pc}
Why would we do this?
\end{frame}

% % %
\begin{frame}{}{}
\[
\includegraphics[scale=0.17]{matroidA.eps}
\]

{\color{powCol}Q:} What do you notice about the points in relation to the matroid $\mathscr M_A$?

\pause
\vspace{1pc}
Points $1$, $2$, and $3$ are collinear if and only if $\{1,2,3\}$ is not a basis for $\mathscr M_A$.  We say the \textbf{bracket} $[123]$ vanishes, or equals 0.  
\end{frame}

% % %
\begin{frame}
The bracket is shorthand for determinant:
\[
[123]=0\iff 
\begin{vmatrix}
x_{11} & x_{12} & x_{13} \\
x_{21} & x_{22} & x_{23} \\
x_{31} & x_{32} & x_{33}
\end{vmatrix}
=0
\]
This is a defining equation in the $x$'s!

\pause
\vspace{1pc}
{\color{powCol}Q:} What is another defining equation for $\Var_A$?

\pause
\vspace{1pc}
Are there others? \pause ???
\end{frame}

% % %
\begin{frame}{}{}
{\color{powCol}Example (Ford 2013):} 

\vspace{0.5pc}
$\mathscr M_{\text{pencil}}=$ matroid on a $3\times 7$ matrix with \emph{non}bases $\{1,2,7\},\{3,4,7\},\{5,6,7\}$  

\vspace{-1pc}
\begin{center}
\includegraphics[scale=0.225]{GEFMVpencil.eps}
\end{center}
{\color{powCol}Q:} Which brackets vanish?
\end{frame}

% % %
\begin{frame}{}{}
\[
\includegraphics[scale=0.35]{ford}
\]
{\color{powCol}All hell breaks loose:} $[134][256]-[234][156]=0$ is also a defining equation for $\Var_{\text{pencil}}$.  \pause How could we have known this?!

\pause
\vspace{1pc}
{\color{powCol}Ans:} The \textbf{Grassmann-Cayley algebra}.  %\pause It lets us produce polynomial equations in the brackets (i.e. determinants).
\end{frame}

% % % % %
\section{Tool: Grassmann-Cayley algebra}

% % % 
\begin{frame}{Tool}{Grassmann-Cayley algebra}
The Grassmann-Cayley algebra was invented to do \emph{synthetic projective geometry}.

\vspace{1pc}
We do arithmetic on the points (vectors) themselves, and obtain expressions in the brackets.

\pause
\vspace{1pc}
There are two operations:
\end{frame}

% % %
\begin{frame}  
\textbf{join} ($\vee$): refers to the line passing through points
\[
\includegraphics[scale=0.35]{GEFMVjoin.eps}
\]
The line joining $1$ and $2$ is $1\vee 2$, or $12$.  

\pause
\vspace{1pc}
Three points joined makes a bracket, and three collinear points make the bracket vanish.
\end{frame}

% % % 
\begin{frame}
\textbf{meet} ($\wedge$): refers to the intersection of two lines 
\[
\includegraphics[scale=0.1]{GEFMVmeet.eps}
\]
The meet of the lines $1\vee 2$ and $3\vee 4$ is $(1\vee 2)\wedge (3\vee 4)$, or $12\wedge 34$.

\pause 
\vspace{1pc}
The meet operation uses \emph{shuffle products}, which also produce expressions in the brackets.
\end{frame}

% % % 
\begin{frame}{}{}
\begin{center}
\includegraphics[scale=0.225]{GEFMVpencil.eps}
\end{center}

We have $((1\vee 2)\wedge(3\vee 4))\vee 5\vee 6=0$, because these three points are collinear.  \pause Here's how to apply the shuffles: \pause
\begin{align*}
(12\wedge 34)\vee 56 &= ([134]2-[234]1)56 \\
 &= [134][256]-[234][156] = 0
\end{align*}
\end{frame}

% % % % %
%\section{Application: Pascal's theorem}

% % %
\begin{frame}{}{}%{Application}{Pascal's theorem}
%\vspace{3.46pc}
{\color{powCol}Example:} \textbf{Pascal's theorem} says if six points on a conic are joined in the way illustrated below, then the resulting intersection points (7, 8, and 9) are collinear.
\begin{center}
\includegraphics[scale=0.25]{GEFMVpascals-theorem.eps}
\end{center}
\end{frame}

% % %
\begin{frame}{}{}
\begin{center}
\includegraphics[scale=0.25]{GEFMVpascals-theorem.eps}
\end{center}

We have a matroid $\mathscr M_{\text{Pascal}}$ corresponding to this configuration of points.  \pause  Let's find defining equations for $\Var_{\text{Pascal}}$.

\pause
\vspace{1pc}
{\color{powCol}Q:} Which brackets vanish?
\end{frame}

% % %
\begin{frame} 
\begin{thm}[Sidman, Traves, W] 
The defining equations for the matroid variety $\Var_{\text{Pascal}}$ include at least one quartic, three independent cubics, and three independent quadrics in the brackets, besides the vanishing brackets
$
[127], [238], [349], [457], [568], [169], [789]$.
\end{thm}
\end{frame}

% % % 
\begin{frame}{}{}
{\color{powCol}Finding the quartic:}
The statement of Pascal's theorem says we have 
$%\[
(12\wedge 45)\vee (23\wedge 56)\vee (34\wedge 61) = 0.
$%\]

\begin{center}
\includegraphics[scale=0.25]{GEFMVpascals-theorem.eps}
\end{center}
\end{frame}

% % % 
\begin{frame}
Apply the shuffle products:
\begin{align*}
& (12\wedge 45)\vee (23\wedge 56)\vee (34\wedge 61) \\
 =&\ ([145]2-[245]1)\vee([256]3-[356]2)\vee([361]4-[461]3) 
\end{align*}
\end{frame}

% % % 
\begin{frame}
Now ``foil":
\begin{align*}
 & {\color{powCol}(}([145]2-[245]1)\vee([256]3-[356]2){\color{powCol})}\vee([361]4-[461]3)  \\ 
 =&\ 
 {\color{powCol}(}[145][256]23-[145][356]22-[245][256]13+[245][356]12{\color{powCol})} \\
 & \quad \vee([361]4-[461]3) 
\end{align*}
\end{frame}

% % %
\begin{frame}
Finish ``foiling": 
\begin{align*}
 & \left([145][256]23-[145][356]22-[245][256]13+[245][356]12\right) \\
 & \quad \vee([361]4-[461]3) \\
 =&\ [145][256][361][234]-[145][256][461][233] \\
 & \quad -[145][356][361][224]+[145][356][461][223] \\
 & \quad -[245][256][361][134]+[245][256][461][133] \\
 & \quad +[245][356][361][124]-[245][356][461][123] 
\end{align*}

\pause
{\color{powCol}Q:} What happens when two numbers share a bracket?
\end{frame}

% % %
\begin{frame}
\[
[233]\leftrightarrow
\begin{vmatrix}
x_{12} & x_{13} & x_{13} \\
x_{22} & x_{23} & x_{23} \\
x_{32} & x_{33} & x_{33}
\end{vmatrix}
\]
{\color{powCol}Q:} What happens to a matrix with two repeated columns?

\pause
\vspace{1pc}
It means we get a bunch of cancelling...
\end{frame}

% % % 
\begin{frame} 
\begin{align*}
 & [145][256][361][234]-\cancel{[145][256][461][233]} \\
 & \quad -\cancel{[145][356][361][224]}+\cancel{[145][356][461][223]} \\
 & \quad -[245][256][361][134]+\cancel{[245][256][461][133]} \\
 & \quad +[245][356][361][124]-[245][356][461][123] \\
 =&\ [145][256][361][234]-[245][256][361][134] \\
 & \quad +[245][356][361][124]-[245][356][461][123]=0
\end{align*}

\pause
The statement of Pascal's theorem gives us a non-obvious defining equation for $\Var_{\text{Pascal}}$!
\end{frame}

% % % 
\begin{frame}{}{}
{\color{powCol} Conclusion:} \pause
\vspace{0.25pc}
\begin{itemize}
\item Finding defining equations for matroid varieties is HARD. \pause
\item The Grassmann-Cayley algebra lets us find \emph{some} of the defining equations, by using the geometry of the matroids when thought of as point configurations.
\end{itemize}
\pause
\vspace{1pc}
\textcolor{powCol}{Further work:} \pause
\vspace{0.25pc}
\begin{itemize}
\item Are our equations the only defining equations for $\Var_{\text{Pascal}}$? \pause
\item Defining equations for matroid varieties given by other point configurations (e.g., Pappus' theorem)? \pause
\item Defining equations for matroid varieties given by wheel graphs?
\end{itemize}
\end{frame}

% % % 
\begin{frame}
\textcolor{powCol}{Q \& A:} \pause
\vspace{0.25pc}
\begin{itemize}
\item[(1)] What do you love most about mathematics? \pause
\item[(2)] What was your favorite math topic as an undergraduate? \pause
\item[(3)] How did you prepare for graduate school? \pause
\item[(4)] What was the most challenging part of grad school and how did you handle it? \pause
\item[(5)] Do you have any general advice for students considering pursuing the doctorate? \pause
\end{itemize}

\vspace{1pc}
\textcolor{powCol}{Contact:} \url{awheeler@mtholyoke.edu}
\pause
\vspace{1pc}
\hrule

%\pause
\vspace{0.25pc}
{\color{powCol}
\begin{flushright}
Thank-you!
\end{flushright}
}
\end{frame}

% % % % %
%\usebackgroundtemplate{
%\includegraphics[width=\paperwidth,height=\paperheight]{JTS2020PascalBkgdHexagon.png}
%}
	
% % %
%\begin{frame}{}{}
%
%\end{frame}
%
%\usebackgroundtemplate{
%\includegraphics[width=\paperwidth,height=\paperheight]{JTS2020PascalBkgd.png}
%}
%	
% % %
%\begin{frame}{}{}
%
%\end{frame}
%
\usebackgroundtemplate{
\includegraphics[width=\paperwidth,height=\paperheight]{JTS2020PascalBkgd.eps}
}

% % % % % % % % % % 
\end{document}