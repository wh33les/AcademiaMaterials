\documentclass[../algebraNotesMSRI-UP2016.tex]{subfiles}

\begin{document}

\section[\S \thesection]{Basic properties of groups}
% % % % %
\subsection[\subsecname]{Identities and inverses}\label{subsec:identitiesInverses}
% % %
\begin{frame}{\subsecname}
In mathematics defining a new concept entails pointing out the more immediate, even obvious, results -- with proof.  These are known as \vocab{propositions}.

\smallGap
\begin{prop}
The identity element of a group $(G,\star)$ is unique.
\end{prop}

\smallGap
\begin{proof}
Suppose $e,e'\in G$ are both identity elements.  Then we have 
\[
e=e\star e'=e'\star e=e'
\] 
so $e=e'$ are the same element.
\end{proof}
\end{frame}

% % %
\begin{frame}[c]
Often, when multiplicative notation is used $1$ denotes the identity element.  Likewise, we use $0$ to denote the identity under additive notation.  

\smallGap
\begin{que} The set $\Z$ has both $0$ and $1$.  Are they equal?  If not, then which is the true identity element?  \textit{Hint: Under which operation is $\Z$ a group?}
\end{que}

\smallGap
We shall see in Section \ref{subsec:cyclicDefs}, that $1$ \vocab{generates} the group $\Z$.  Meanwhile, $0$ generates the \vocab{trivial} subgroup of $\Z$.
\end{frame}

% % %
\begin{frame}[c]
\begin{prop}[The Cancellation Law]
Let $(G,\star)$ denote a group.  For every $a,b,c\in G$, 
\[
a\star b=c\star b\implies a=c\qquad\text{ and }\qquad b\star a=b\star c\implies a=c.
\]
\end{prop}

\smallGap
The $\implies$ symbol means ``implies".


\smallGap
\begin{exe}\label{exe:cancellationLaw}
Prove the Cancellation Law.
\end{exe}
\end{frame}

% % %
\begin{frame}[c]
To ease readability, in the following we shall use multiplicative notation, i.e, if $g,h$ are elements in the group $G=(G,\star)$, then we write $gh=g\star h$.

\smallGap
\begin{prop}[Properties of Inverses]\label{prop:inverses}
Let $G$ denote a group. 
\begin{enumerate}[(a)]
\item For every $g\in G$, the inverse $g\1$ is unique.
\item For every $g\in G$, $(g\1)\1=g$.
\item (\vocab{Shoes-Socks Theorem})  For every $g,h\in G$, $(gh)\1=h\1g\1$.
\end{enumerate}
\end{prop}

\smallGap
\begin{exe}[cf. Problem 45]\label{exe:inverses}
Prove Proposition \ref{prop:inverses}.
\end{exe}
\end{frame}

% % %
\begin{frame}[c]
\begin{exe}[cf. Problem 46]\label{exe:abelian}
Suppose $G$ is a group with the property that every element is its own inverse.  Prove $G$ must be abelian.
\end{exe}

\smallGap
\begin{que}
What is an example where the \vocab{converse} in Exercise \ref{exe:abelian} fails?  In other words, name an abelian group with an element that is not its own identity.
\end{que}
\end{frame}

% % % % %
\subsection[\subsecname]{Integer exponents}\label{subsec:integerExponents}
% % %
\begin{frame}[c]{\subsecname}
\textbf{Recall:} In $\R$ we have the property that multiplying exponents with the same base is the same as adding the exponents.  In other words, for $a\in \R$, and integers $n,m$, we have 
\begin{equation}\label{eq:expRule1}
a^na^m=a^{n+m}.
\end{equation}

\smallGap
Consequently, we also have the property that
\begin{equation}\label{eq:expRule2}
(a^n)^m=\underbrace{a^n\cdots a^n}_{\text{$m$ times}}=a^{\overbrace{\scriptstyle n+\cdots+n}^{\scriptstyle\text{$m$ times}}}=a^{m\cdot n}.
\end{equation}
\end{frame}

% % %
\begin{frame}[c]
Suppose $g$ is in the group $G=(G,\star)$ and $n\in\N$.  We define $g^1:=g$, then recursively, $g^n:=g^{n-1}\star g^1=g^{n-1}g$.  By associativity of groups, $g^n=\underbrace{g\cdot\cdots \cdot g}_{\text{$n$ times}}$.

\smallGap
\begin{que}
What are the analogous notions when we use \vocab{additive notation}, i.e., for $g,h$ in the group $G=(G,\star)$ write $g+h=g\star h$?
\end{que}

\smallGap
\begin{exe}\label{exe:expRules1and2}
Prove Equations \eqref{eq:expRule1} and \eqref{eq:expRule2} hold for groups.
\end{exe}
\end{frame}

% % % 
\begin{frame}[c]
In the reals we also had a notion of zero and negative exponents, namely, for $a\in\R$ and a positive integer $n$,
\[
a^0=1\qquad\text{ and }\qquad a^{-n}=\frac{1}{a^n}.
\]
We shall generalize this notion for groups.
\end{frame}

% % %
\begin{frame}[c]
Again, use multiplicative notation for the group $G$.  To stay consistent with the definition of a group and with our definition of positive exponents, we must have 
\[
g^0g = g^0g^1 = g^{0+1}=g =eg,
\]
where $e$ is the identity element.  We can right-cancel the $g$ in $g^0g=eg$ to conclude $g^0=e$.  This is why under multiplicative notation we often use $1$ to denote $e$.   

\smallGap
\end{frame}

% % % 
\begin{frame}[c]
Finally, for fixed $n\in\N$ we define $a^{-n}:=(a\1)^n$.  

\smallGap
\begin{UNABexe}[cf. Problem 48]\label{exe:negExp}
Show that indeed this definition of $a^{-n}$ is consistent with the rules of exponents.  In other words, show Equations \eqref{eq:expRule1} and \eqref{eq:expRule2} still work for negative exponents.
\end{UNABexe}

\smallGap
\begin{que}
What are the analogous notions under additive notation?
\end{que}

\smallGap
\begin{exe}\label{exe:negExpD4}
What are the negative exponents for each element in $D_4$, the dihedral group of order 4?  (See Example \ref{ex:D4} and Exercise \ref{exe:prob40}.)
\end{exe}
\end{frame}

% % % % %
\answerKey
% % % 
\begin{frame}{\subsecname}
\exeSol{exe:cancellationLaw}
%\textbf{Proof.}
We manipulate each of the equations $a\star b=c\star b$ and $b\star a=b\star c$ using inverses, associativity, and the identity element, $e$:
\begin{columns}
\begin{column}{0.2\textwidth}
	\begin{align*}
	a\star b &= c\star b \\
	(a\star b)\star b\1 &= (c\star b)\star b\1 \\
	a\star (b\star b\1) &= c\star(b\star b\1) \\
	a\star e &= c\star e \\
	a &= c
	\end{align*}
\end{column}
\begin{column}{0.2\textwidth}
	\begin{align*}
	b\star a &= b\star c \\
	b\1\star(b\star a) &= b\1\star(b\star c) \\
	(b\1\star b)\star a &= (b\1\star b)\star c \\
	e\star a &= e\star c \\
	a &= c
	\end{align*}
\end{column}
\end{columns}	
\qed
\end{frame}

% % %
\begin{frame}
\exeSol[(cf. Problem 45)]{exe:inverses}
%\textbf{Proof.}
\begin{itemize}
\item[(a)]\label{exeSol:inverses-a} Suppose $g\1$ and $\gamma$ are inverses for $g$.  Multiplying either of them by $g$, on either the left or the right, gives the identity:
\begin{align*}
gg\1 &= g\gamma \\
g\1gg\1 &= g\1 g\gamma \\
eg\1 &= e\gamma \\
g\1 &= \gamma 
\end{align*}
and so there is only one inverse.
\item[(b)] Multiplying an element by its inverse gives the identity.  Then multiply on the left by $g$:
\begin{align*}
g\1(g\1)\1 &= e \\
gg\1(g\1)\1 &= ge \\
= (g\1)\1 &= g. 
\end{align*}
\end{itemize}
\end{frame}

% % %
\begin{frame}
\begin{itemize}
\item[(c)] By part {\usebeamercolor[fg]{block title}(a)}, the inverse element is unique.  We just check if multiplying by $h\1g\1$ on either side gives the identity:
\begin{align*}
(gh)(h\1g\1) &= g(hh\1)g\1 \\
	&= geg\1 \\
	&= gg\1 \\
	&= e; \\
(h\1g\1)(gh) &= h\1(g\1g)h \\
		&= h\1eh \\
		&= h\1h \\
		&= e.
\end{align*}
\end{itemize}
\qed
\end{frame}

% % %
\begin{frame}
\exeSol[(cf. Problem 46)]{exe:abelian}
%\textbf{Proof.} 
Choose $g,h\in G$.  We will show $gh=hg$.  Since $g\1=g$ and $h\1=h$, and in particular, $(gh)\1=gh$,
\begin{align*}
(gh)(gh) &= e \\
ghghh &= eh \\
ghg &= h \\
ghgg &= hg \\
gh &= hg.
\end{align*}
\qed
\end{frame}

% % %
\begin{frame}
\exeSol[(cf. Problem 47)]{exe:expRules1and2}
%\textbf{Proof:} 
Let $(G,\star)$ denote a group and choose $g\in G$.  From the recursive definition of positive exponents of $g$,
\begin{align*}
g^m\star g^n &= (g^{m-1}\star g)\star(g^{n-1}\star g) \\
 &= \left((g^{m-2}\star g) \star g\right)\left((g^{n-2}\star g)\star g\right) \\[-5pt]
 &\;\;\vdots \\[-5pt]
 &= (g^1\star \underbrace{g\star \cdots \star g}_{\text{$m-1$ times}})\star (g^1\star \underbrace{g\star \cdots \star g}_{\text{$n-1$ times}}) \\
 &=g^1\star \underbrace{g\star \cdots \star g}_{\text{$m+n-1$ times}} \\
 &=g^{m+n}, \tag{1.1} \\[2pt]
\text{and hence,}\quad(g^m)^n &= \underbrace{g^m\star \cdots \star g^m}_{\text{$n$ times}} \\
	&= \underbrace{g\star \cdots \star g}_{\text{$mn$ times}} = g^{mn}. \tag{1.2}
\end{align*}
\qed
\end{frame}

% % %
\begin{frame}
\begin{block}{Exercise* (cf. Problem 48)}
\end{block}
\vspace{-0.75pc}
\textbf{Solution:}
Let $(G,\star)$ denote a group and choose $g\in G$, $m,n\in\N$.  Then
\begin{align*}
g^{-m}\star g^{-n} &= (g\1)^m\star (g\1)^n \\
	&= (g\1)^{m+n} \\
	&= g^{-(m+n)} \\
	&= g^{-m+(-n)} \tag{1.1} \\
\text{and,}\quad (g^{-m})^{-n} &= ((g\1)^m)^{-n} \\
	&= \left(\left((g\1)^m\right)\1\right)^n.
\end{align*}
To move forward, we claim for any $h\in G$, $h^{-k}=(h^k)\1$.  If true, then when we multiply on either side by $h^k$ we get the identity:
\begin{align*}
(h^k)\star (h^{-k}) &= (h^k)\star (h\1)^k \\
	&= h^{k-1}\star (h\star h\1)\star (h\1)^k \\
	&= h^{k-1}\star e \star (h\1)^{k-1} \\
	&= h^{k-2}\star (h\star h\1)\star (h\1)^{k-2} \\[-5pt]
	&\;\;\vdots \\[-5pt]
	&= e.
\end{align*}
\end{frame}

% % %
\begin{frame}
Likewise,
\begin{align*}
(h^{-k})\star h^k &= (h\1)^k\star h^k \\
	&= (h\1)^{k-1}\star (h\1\star h)\star h^{k-1} \\
	&= (h\1)^{k-1}\star e \star h^{k-1} \\
	&= (h\1)^{k-2}\star (h\1\star h)\star h^{k-2} \\[-5pt]
	&\;\;\vdots \\[-5pt]
	&= e.
\end{align*}
Consequently,
\begin{align*}	
 \left(\left((g\1)^m\right)\1\right)^n &= \left((g\1)^{-m}\right)^n \\
	&= \left((g\1)\1)^m\right)^n \\
	&= (g^m)^n \\
	&= g^{mn} \\[-1pt]
	&= g^{(-m)(-n)}. \tag{1.2}
\end{align*}
\qed
\end{frame}

% % %
\begin{frame}
\exeSol{exe:negExpD4}
%\textbf{Solution:} 
In $D_4$, all negative exponents of $e$ are equal to $e$.  For the other elements, since the positive exponents are cyclic, so are the negative exponents.
\begin{tabular}{p{0.4\textwidth}p{0.4\textwidth}}
{\begin{align*}
r\1 &= r^3 \\
r^{-2} &= (r^2)\1 = r^2 \\
r^{-3} &= (r^3)\1 = r \\
r^{-4} &= (r^4)\1 = e \\
	&\;\;\vdots
\end{align*}
} & {\begin{align*}
s\1 &= s \\
s^{-2} &= (s^2)=e \\
(rs)\1 &= r\1s\1 =r^3s \\
(rs)^{-2} &= (r^3s)^2 = e \\
(r^2s)\1 &= r^2s \\
(r^2s)^{-2} &= (r^2s)^2=e \\
(r^3s)\1 &= rs \\
(r^3s)^{-2} &= (rs)^2 = e
\end{align*}
}
\end{tabular}
\qed
\end{frame}

% % % % % % % % % % % % % % % % % % % %
\end{document}