\documentclass[../algebraNotesMSRI-UP2016.tex]{subfiles}

\begin{document}

\section[\S \thesection]{Fundamental Theorem of Finitely Generated Abelian Groups}\label{sec:2p10FTOFinitelyGeneratedAbelianGroups}
% % % % %
\subsection[\subsecname]{Presentation of a group}
% % %
\begin{frame}{\subsecname}{}
\begin{dfn}
Suppose $\varphi:\Z^n\to \Z^m$ is a homomorphism.
\begin{enumerate}[(a)]
\item The \vocab{cokernel} of $\varphi$ is the quotient group $\coker{(\varphi)}:=\Z^m/	\image{(\varphi)}$.
\item Suppose $\varphi$ is given by an $m\times n$ matrix $A$.  Write $A\Z^n\norml \Z^m$ to denote $\image{(\varphi)}$.  Any isomorphism
\[
\psi: \Z^m/A\Z^n\overset{\cong}{\to} G
\]
is called a \vocab{presentation} of a finitely generated abelian group $G$, and $A$ is called a \vocab{presentation matrix} for $G$.
\end{enumerate}
\end{dfn}
%{\bf REMARK ABOUT HILBERT BASES?}
\end{frame}

% % %
\begin{frame}[c]{}{}
\begin{exe}\label{exe:kernelsAndCokernels}
Let $\varphi:G\to H$ denote a group homomorphism.  Prove the following:
\begin{enumerate}[(a)]
\item $\varphi$ is injective if and only if $\ker{\varphi}=\{1_G\}$.
\item $\varphi$ is surjective if and only if $\coker{\varphi}=\{1_H\}$.
\end{enumerate}
\end{exe}
\end{frame}

% % %
\begin{frame}{}{}
\begin{ex}\label{ex:notObviousZmod5Z}
Suppose $\varphi:\Z^2\to \Z^2$ is a group homomorphism given by the matrix 
\[
A=\begin{pmatrix*}[r]
	2 & -1 \\
	1 & 2
	\end{pmatrix*}.
\]
We wish to find $G$ such that there is an isomorphism
\[
\psi:\Z^2/A\Z^2\xrightarrow{\cong} G.
\]
\end{ex}

\smallGap
Looking at the entire composition 
\begin{equation}\label{eq:compPresentation}
\Psi:\Z^2\overset{\bar{\varphi}}{\surj} \Z^2/A\Z^2 \overset{\psi}{\to} G,
\end{equation}
where $\bar{\varphi}$ denotes the composition of $\varphi$ with the natural map $\Z^2\to\Z^2/\ker{(\varphi)}=\Z^2/A\Z^2$; should such an isomorphism $\psi$ exist then $g_1:=\Psi(\vect e_1)$ and $g_2:=\Psi(\vect e_2)$ must generate $G$ (or else $\psi$ would not be surjective).
\end{frame}

% % %
\begin{frame}[c]
By construction, we must have $\ker{\Psi}=A\Z^2$.  In other words,
\begin{align*}
A\vect e_1= 
	\begin{pmatrix*}[r]
	2 & -1 \\
	1 & 2 
	\end{pmatrix*}
	\begin{pmatrix}
	1 \\
	0 
	\end{pmatrix}	
	&= \begin{pmatrix}
		2 \\
		1
		\end{pmatrix}
	\mapsto 0 \quad\text{ and } \\
A\vect e_2= 
	\begin{pmatrix*}[r]
	2 & -1 \\
	1 & 2 
	\end{pmatrix*}
	\begin{pmatrix}
	0 \\
	1 
	\end{pmatrix}	
	&= \begin{pmatrix}
		-1 \\
		2
		\end{pmatrix}
	\mapsto 0.	
\end{align*}

\smallGap
For $\Psi$ to be a group homomorphism, we must have  
\begin{align*}
\Psi(2\vect e_1)+\Psi(\vect e_2) &= 0 = \Psi(-1\vect e_1)+\Psi(2\vect e_2) \\
\implies 2g_1+g_2 &= 0 = -g_1+2g_2.
\end{align*}
\end{frame}

% % %
\begin{frame}[c]
We get a system of 2 equations with 2 unknowns; solving it, 
\begin{align}
\left\{\begin{array}{rl}\label{eq:g2}
	2g_1+g_2=0 & \implies \boxed{g_2=-2g_1} \\
	-g_1+2g_2= 0 &
	\end{array}\right. \\
\begin{aligned}\label{eq:g1}
	\implies -g_1+2g_2 &= -g_1+2(-2g_1) = 0	\\
		&= -5g_1=\boxed{0=5g_1}. 	
	\end{aligned}
\end{align}
\end{frame}

% % %
\begin{frame}[c]{}{}
Remember, working over $\Z$ (versus $\R$), we are only allowed to divide by $\pm 1$.  The the boxed equations in \eqref{eq:g2}, \eqref{eq:g1} cannot be simplified any further.  

\smallGap
$G$ is generated by  
\[
\langle g_1,g_2\rangle = \langle g_1,-2g_1\rangle = \langle g_1\rangle = G,
\]
i.e., $G$ is cyclic with generator $g_1$.  The condition $5g_1=0$ allows a direct isomorphism
\begin{align*}
G &\overset{\cong}{\to} \Z/5\Z \\
g_1 &\mapsto 1.
\end{align*}
\end{frame}

% % %
\begin{frame}[c] 
\begin{ex}
A ``better" presentation for $\Z/5Z$ is given by $A=5$ (a $1\times 1$ integer matrix is just an integer).  Left multiplication by $A$ is a homomorphism  
\begin{align*}
\varphi: \Z^1 &\to \Z^1 \\
 n &\mapsto 5n
\end{align*}  
whose cokernel is $\Z^1/\image{(\varphi)}=\Z/5\Z$.
\end{ex}
\end{frame}

% % % % %
\subsection[\subsecname]{Row and column operations}
% % %
\begin{frame}{\subsecname}{}
The key step in Example \ref{ex:notObviousZmod5Z} was solving the system of equations
\[
\left\{\begin{aligned}
	2g_1+g_2 &= 0 \\
	-g_1+2g_2 &= 0
	\end{aligned}
	\right.
\]
or, equivalently, performing the following row and column operations
\[\footnotesize
\begin{pmatrix*}[r]
2 & 1 \\
-1 & 2
\end{pmatrix*} \xrightarrow{C_1\mapsto C_1-2C_2}
\begin{pmatrix*}[r]
0 & 1 \\
-5 & 2 
\end{pmatrix*}   \xrightarrow{R_2\mapsto R_2-2R_1} 
\begin{pmatrix*}[r]
0 & 1 \\
-5 & 0 
\end{pmatrix*} \xrightarrow{R_2\mapsto -R_2}
\begin{pmatrix}
0 & 1 \\
5 & 0 
\end{pmatrix}.
\]
Then of course, if we want to, we can interchange columns to standardize the process.  
\[
\begin{pmatrix}
0 & 1 \\
5 & 0 
\end{pmatrix} \xrightarrow{C_1\leftrightarrow C_2}
\begin{pmatrix}
1 & 0 \\
0 & 5 
\end{pmatrix}
\]
\end{frame}

% % %
\begin{frame}[c]
The point is, killing the subgroup $A\Z^2< \Z^2$ is equivalent to killing each of the subgroups $\Z< \Z$ and $5\Z<\Z$, and then taking the direct product:
\[
\Z^2/A\Z^2 \cong \Z/\Z\times \Z/5\Z \cong \Z/5\Z. 
\]
(The last isomorphism is projection from $\{0\}\times \Z/5\Z\to \Z/5\Z$.)
\end{frame}

% % %
\begin{frame}
\begin{prop}\label{prop:allOperations}
Suppose $A$ is an $m\times n$ presentation matrix for a finitely generated abelian group $G$.  Then any of the following operations will result in a presentation matrix for $G$:
\begin{enumerate}[(i)]
\item add an integer multiple of one column (resp. row) to another;
\item interchange two columns (resp. rows);
\item multiply a column (resp. row) by $\pm 1$; 
\item delete a column 
of zeros;
\item delete the $i$th row and $j$th column, \textbf{provided} the $j$th column is $\vect e_i$. %{\bf A LITTLE EXPLANATION WHY, AND FOR PART 4}
\end{enumerate}
\end{prop}
\end{frame}

% % %
\begin{frame}
\bigProof
Write $\psi:\Z^m/A\Z^n\overset{\cong}{\to}G$ and let the vectors $\vect a_1,\dots,\vect a_n$ denote the respectively indexed columns of $A$.  Since $\ker{\psi}=A\Z^n$, its generators are $\vect a_1,\dots, \vect a_n$.

\smallGap 
Operations {\usebeamercolor[fg]{block title}(i)}-{\usebeamercolor[fg]{block title}(iii)} are permissible by Theorem \ref{thm:SmithNormalForm}.  

\smallGap
We first prove {\usebeamercolor[fg]{block title}(iv)}.  Attaining a matrix with a column consisting of zeros can be done using only operations {\usebeamercolor[fg]{block title}(i)}-{\usebeamercolor[fg]{block title}(iii)}.  So \vocab{wolog} (without loss of generality), assert the $j$th column of $A$ consists of zeros.  Deleting $\vect a_j=\vect 0$ does not change the kernel, but it does produce an $m\times (n-1)$ matrix $A'$ defining a homomorphism $\varphi':\Z^{n-1}\to \Z^m$.  The kernel of $\psi$ is unchanged means $A'\Z^{n-1}=A\Z^n$ and hence $\psi: \Z^m/A\Z^n=\Z^m/A'\Z^{n-1}\overset{\cong}{\to} G$.
\end{frame}

% % %
\begin{frame}[c]
To prove {\usebeamercolor[fg]{block title}(v)}, again, by the operations {\usebeamercolor[fg]{block title}(i)}-{\usebeamercolor[fg]{block title}(iii)} it suffices to suppose the $j$th column of $A$ is the $i$th unit vector, i.e., $\vect a_j=\vect e_i$.  Let $\Psi:\Z^n\to G$ denote the composition as in Equation \eqref{eq:compPresentation} (with 2 replaced by $n$).  The images of the standard basis vectors under $\Psi$ generate $G$; all vectors $\vect w\in A\Z^n$ have zeros in the $i$th entry, which is determined by the $i$th row of $A$ so we omit it.  Likewise, $\Psi({\vect e_i})=0$ means we can omit $\vect e_i\in \Z^n$.
\qed
\end{frame}

% % %
\begin{frame}{}{}
\begin{ex}
Proposition \ref{prop:allOperations} speeds up the process of finding a group presentation.  For example,
\begin{multline*}
A=\begin{pmatrix}
	3 & 8 & 7 & 9 \\
	2 & 4 & 6 & 6 \\
	1 & 2 & 2 & 1
\end{pmatrix}\xrightarrow{\substack{R_1\to R_1-3R_3 \\ R_2\to R_2-2R_3}}
	\begin{pmatrix}
	0 & 2 & 1 & 6 \\
	0 & 0 & 2 & 4 \\
	1 & 2 & 2 & 1
	\end{pmatrix}\xrightarrow{\xcancel{C_1}\quad \xcancel{R_3}}
		\begin{pmatrix}
		2 & 1 & 6 \\
		0 & 2 & 4
		\end{pmatrix} \\
\xrightarrow{R_2\to R_2-2R_1}
	\begin{pmatrix*}[r]
	2 & 1 & 6 \\
	-4 & 0 & -8
	\end{pmatrix*}\xrightarrow{\xcancel{C_2}\quad \xcancel{R_1} }
		\begin{pmatrix}
		-4 & -8
		\end{pmatrix}  \\
\xrightarrow{C_2\to C_2+2C_1}
	\begin{pmatrix}
	-4 & 0 
	\end{pmatrix}\xrightarrow{\substack{\xcancel{C_2} \\ C_1\to -C_1}}
		\begin{pmatrix}
		4
		\end{pmatrix}.		
\end{multline*}
So $(4)\Z$ is a $1\times 1$ matrix defining a homomorphism $\varphi:\Z^1\to\Z^1$.  $A$ represents its cokernel, $\Z/(4)\Z=\Z/4\Z$. 
\end{ex}
\end{frame}

% % %
\begin{frame}[c]{}{}
Here we mention, but without proof, a key ingredient in proving the Fundamental Theorem of Finitely Generated Abelian Groups:

\smallGap
\begin{prop}
Suppose $G$ is a free abelian group of rank $m$, and $H<G$.  Then $H$ is free abelian of rank at most $m$.
\qed
\end{prop}
\end{frame}

% % %
\begin{frame}[c]{}{}
\begin{thm}[Fundamental Theorem of Finitely Generated Abelian Groups]\label{thm:ftofgab}
Let $G$ denote a finitely generated abelian group.  Then there exist positive integers $d_1,\dots,d_k$ and non-negative integer $r\geq 0$, with $d_1|\cdots|d_k$ such that
\[
G\cong\Z_{d_1}\times\cdots\times\Z_{d_k}\times\Z^r.
\]
The integers $d_1,\dots,d_k$ are uniquely determined and called \vocab{invariant} factors.  The integer $r$ is also unique and is called the \vocab{free rank of $G$}.
\end{thm}
\end{frame}

% % % % %
\subsection[\subsecname]{Applications}
% % %
\begin{frame}[c]{\subsecname}{}
\begin{ex}
Recall, in Example \ref{ex:diagonalAlgorithm} we reduced the matrix $A$ to its Smith normal form. 
\[
A=\begin{pmatrix*}[r]
	1 & -1 & 1 \\
	5 & 1 & -5 \\
	-3 & -3 & 29
	\end{pmatrix*}\to 
\begin{pmatrix}
1 & 0 & 0 \\
0 & 2 & 0 \\
0 & 0 & 66
\end{pmatrix}.
\]
By Proposition \ref{prop:allOperations}, we can reduce further to $\left(\begin{smallmatrix}
	2 & 0 \\
	0 & 66
	\end{smallmatrix}\right)$.  Thus $A$ is a presentation matrix for $G=\Z/2\Z\times \Z/66\Z$.
\end{ex}
\end{frame}

% % %
\begin{frame}[c]
\begin{exe}[cf. Problem 79]\label{exe:prob79}
What direct product of cyclic groups is presented by the matrix $\begin{pmatrix}
2 & 1 \\
1 & 2
\end{pmatrix}$? 
\end{exe}

\smallGap
\textbf{Note:} In the spirit of keeping additive notation, from now on we shall refer to direct products of groups as \vocab{direct sums}, with the notation $\oplus$ instead of $\times$.  In practice there is a difference between the two symbols, but only in the case of infinite sums/products.
\end{frame}

% % %
\begin{frame}[c]{}{}
\begin{exe}[cf. Problem 80]\label{exe:prob80}
Use your computations from Problem \ref{exe:prob77} to find a direct sum of cyclic groups presented by the transposed Laplacian of the graph in Figure \ref{fig:graphWithLoops}.
\end{exe}
\end{frame}

% % %
\begin{frame}{}{}
\begin{exe}[cf. Problem 81]\label{exe:prob81}
Compute, by hand, a direct sum of cyclic groups isomorphic to the abelian groups presented by the following matrices:

\begin{enumerate}[(a)]
\item $\begin{pmatrix}
	5 & 0 & 0 
	\end{pmatrix}$,
\item $\begin{pmatrix}
	5 \\
	0 \\
	0
	\end{pmatrix}$,
\item $\begin{pmatrix}
	2 & 2 & 2 \\
	2 & 2 & 0 \\
	2 & 0 & 2
	\end{pmatrix}$.
\end{enumerate}	
\end{exe}
\end{frame}

% % %
\begin{frame}{}{}
\begin{exe}[cf. Problem 82]\label{exe:prob82}
Compute, using Sage, a direct sum of cyclic groups isomorphic to the abelian groups presented by the following matrices
\begin{enumerate}[(a)]
\item $\begin{pmatrix*}[r]
	3 & -1 & -1 \\
	-1 & 3 & -1 \\
	-1 & -1 & 3
	\end{pmatrix*}$
\item $\begin{pmatrix*}[r]
	3 & -1 & 0 & -1 \\
	-1 & 3 & -1 & 0 \\
	0 & -1 & 3 & -1 \\
	-1 & 0 & -1 & 3
	\end{pmatrix*}$	
\item $\begin{pmatrix*}[r]
	3 & -1 & 0 & 0 & -1 \\
	-1 & 3 & -1 & 0 & 0 \\
	0 & -1 & 3 & -1 & 0 \\
	0 & 0 & -1 & 3 & -1 \\
	-1 & 0 & 0 & -1 & 3
	\end{pmatrix*}$
\end{enumerate}
\end{exe}
\end{frame}

% % %
\begin{frame}[c]
\begin{que}
Compute the determinants for the matrices in Exercises \ref{exe:prob81} and \ref{exe:prob82}.  What is the pattern?
\end{que}
\end{frame}

% % %
\begin{frame}
\begin{exe}[cf. Problem 83]\label{exe:prob83}
For any positive integer $n$, consider an $n\times n$ matrix $A_n$ described by Pascal's triangle, exemplified by 
\[
A_5=
\begin{pmatrix}
1 & 1 & 1 & 1 & 1 \\
1 & 2 & 3 & 4 & 5 \\
1 & 3 & 6 & 10 & 15 \\
1 & 4 & 10 & 20 & 35 \\
1 & 5 & 15 & 35 & 70
\end{pmatrix}.
\]
What finitely generated abelian group $G_n$ is presented by the matrix $A_n$?
\end{exe}
\end{frame}

% % % % %
\answerKey
% % %
\begin{frame}{\subsecname}
\exeSol{exe:kernelsAndCokernels}
\begin{itemize}
\item[(a)] First, suppose $\varphi$ is injective.  Since $1_G\in \ker\varphi$, if $g\in \ker\varphi$ then $\varphi(g)=1_H=\varphi(1_G)$ implies $g=1_G$.  On the other hand, if the kernel is trivial then suppose $\varphi(g)=\varphi(h)$.  Multiply both sides by $\varphi(g)\1$:
\begin{align*}
\varphi(g) &= \varphi(h) \\
\varphi(g)\varphi(g)\1 &= \varphi(h)\varphi(g)\1 \\
1_H &= \varphi(hg\1).
\end{align*}
This means $hg\1 \in\ker\varphi$ so $hg\1=1_G$.  Multiply both sides by $g$:
\begin{align*}
hg\1 &= 1_G \\
hg\1g &= 1_Gg \\
h &= g, 
\end{align*}
and so $\varphi$ is injective.
\qed 
\end{itemize}
\end{frame}

% % %
\begin{frame}[c]
\begin{itemize}
\item[(b)] Say $\varphi$ is surjective.  Then $\varphi(G)=H$ implies $\coker\varphi=H/\varphi(G)=H/H=\{1_H\}$.  Conversely, suppose the cokernel is trivial.  By Lagrange's Theorem, the order of the subgroup $\varphi(G)$ must divide $|H|$ and since the cosets of a subgroup partition the group evenly, we must have 
\[
|H/\varphi(G)|\cdot |\varphi(G)|=|H|.
\]
A trivial cokernel means $|H/\varphi(G)|=1$ and it follows that $|\varphi(G)|=|H|$ and hence, $\varphi(G)=H$.  Therefore, $\varphi$ is surjective.
\qed
\end{itemize}
\end{frame}

% % %
\begin{frame}[c]
\exeSol[(cf. Problem 79)]{exe:prob79}
The Smith normal form of $A:=\left(\begin{smallmatrix} 2 & 1 \\ 1 & 2\end{smallmatrix}\right)$ is $\left(\begin{smallmatrix} 1 & 0 \\ 0 & 3 \end{smallmatrix}\right)$, which means $A$ is a presentation matrix for a group isomorphic to $\boxed{\Z_3}$.

\smallGap
\exeSol[(cf. Problem 80)]{exe:prob80}
The Smith normal form of $\tilde{\Delta}$ from Problem \ref{exe:prob77} was $\left(\begin{smallmatrix} 1 & 0 \\ 0 & 4 \end{smallmatrix}\right)$.  Thus $\mathscr S(\Gamma)\cong \boxed{\Z_4}$.
\end{frame}

% % %
\begin{frame}
\exeSol[(cf. Problem 81)]{exe:prob81}
\begin{itemize}
\item[(a)] $\begin{pmatrix}
	5 & 0 & 0 
	\end{pmatrix}\xrightarrow{\xcancel{C_2} \quad \xcancel{C_3}} (5)$ gives a presentation for a group isomorphic to $\boxed{\Z_5}$.

\smallGap
\item[(b)] $\begin{pmatrix}
5 \\
0 \\
0
\end{pmatrix}$ has $5$ as an invariant factor, and free rank of $2$.  Therefore its cokernel is isomorphic to $\boxed{\Z_5\oplus\Z\oplus\Z}$.

\smallGap
\item[(c)] Computing the Smith normal form 
\begin{multline*}\begin{pmatrix}
	2 & 2 & 2 \\
	2 & 2 & 0 \\
	2 & 0 & 2
	\end{pmatrix} \xrightarrow{\substack{R_2\to R_2-R_1 \\ R_3\to R_3-R_1}}
\begin{pmatrix*}[r]
2 & 2 & 2 \\
0 & 0 & -2 \\
0 & -2 & 0 
\end{pmatrix*} \xrightarrow{-R_2\leftrightarrow -R_3} \\
\begin{pmatrix}
2 & 2 & 2 \\
0 & 2 & 0 \\
0 & 0 & 2
\end{pmatrix} \xrightarrow{\substack{C_2\to C_2-C_1 \\ C_3\to C_3-C_1  }}
\begin{pmatrix}
2 & 0 & 0 \\
0 & 2 & 0 \\
0 & 0 & 2
\end{pmatrix}	
\end{multline*}
shows the abelian group presented by the above matrix is isomorphic to $\boxed{\Z_2\oplus \Z_2 \oplus \Z_2}$.
\end{itemize}
\end{frame}

% % %
\begin{frame}
\exeSol[(cf. Problem 82)]{exe:prob82}
Via Sage, the Smith normal forms reveal the isomorphic abelian groups:
\begin{itemize}
\item[(a)] $\begin{pmatrix*}[r]
	3 & -1 & -1 \\
	-1 & 3 & -1 \\
	-1 & -1 & 3
	\end{pmatrix*}\to 
	\begin{pmatrix}
	1 & 0 & 0 \\
	0 & 4 & 0 \\
	0 & 0 & 4
	\end{pmatrix}$ has cokernel $\boxed{\Z_4\oplus \Z_4}$.
	
\smallGap	
\item[(b)] $\begin{pmatrix*}[r]
	3 & -1 & 0 & -1 \\
	-1 & 3 & -1 & 0 \\
	0 & -1 & 3 & -1 \\
	-1 & 0 & -1 & 3
	\end{pmatrix*}\to
	\begin{pmatrix}
	1 & 0 & 0 & 0 \\
	0 & 1 & 0 & 0 \\
	0 & 0 & 3 & 0 \\
	0 & 0 & 0 & 5
	\end{pmatrix}$ has cokernel $\boxed{\Z_3\oplus \Z_{15}}$.	

\smallGap
\item[(c)] $\begin{pmatrix*}[r]
	3 & -1 & 0 & 0 & -1 \\
	-1 & 3 & -1 & 0 & 0 \\
	0 & -1 & 3 & -1 & 0 \\
	0 & 0 & -1 & 3 & -1 \\
	-1 & 0 & 0 & -1 & 3
	\end{pmatrix*}\to
	\begin{pmatrix}
	1 & 0 & 0 & 0 & 0 \\
	0 & 1 & 0 & 0 & 0 \\
	0 & 0 & 1 & 0 & 0 \\
	0 & 0 & 0 & 11 & 0 \\
	0 & 0 & 0 & 0 & 11
	\end{pmatrix}$ 
	
	has cokernel $\boxed{\Z_{11}\oplus \Z_{11}}$.
\end{itemize}
\end{frame}

% % %
\begin{frame}
\exeSol[(cf. Problem 83)]{exe:prob83}
When $n=1$, the trivial group is presented.  Suppose, for induction, the trivial group is presented by $A_{n-1}$.  To clear the first column of $A_n$, subtract from each row $R_i$, for $i=2,\dots,n$, the row $R_{i-1}$.  For $n\geq 2$ every entry $a_{ij}$ not in the first row or column can be written 
\[
a_{ij}=a_{i,j-1}+a_{i-1,j}.
\]
Thus upon clearing the first column, such an entry becomes $a_{ij}-a_{i-1,j}=a_{i,j-1}$.  

\smallGap
Next, clear the first row by subtracting from each column $C_j$, for $j=2,\dots,n$, the column $C_{j-1}$.  For $i,j\geq 2$, the $i$th entry in the $j$th column was replaced in the row operations by $a_{i,j-1}$.  Thus when we clear the first row the $(i,j)$th entry, for $i,j\geq 2$, will become 
\begin{align*}
a_{i,j-1}-a_{i,j-2} &= (a_{i-1,j-1}+a_{i,(j-1)-1})-a_{i,j-2} \\
	&= a_{i-1,j-1}.
\end{align*}

Having cleared the first row and the first column of $A_n$, the remaining $(n-1)\times (n-1)$ submatrix is a copy of $A_{n-1}$ which, by the induction hypothesis, reduces to the identity matrix.  It follows that $A_n$ also reduces to the identity matrix, and therefore is a presentation for the trivial group.
\qed
\end{frame}

% % % % % % % % % %
\end{document}