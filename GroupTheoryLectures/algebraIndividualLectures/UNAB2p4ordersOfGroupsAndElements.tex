\documentclass[../algebraNotesMSRI-UP2016.tex]{subfiles}

\begin{document}

\section[\S \thesection]{Orders of groups and elements}
% % % % %
\subsection[\subsecname]{Order of an element}
% % %
\begin{frame}{\subsecname}
\begin{dfn}
Let $G$ denote a group with $g\in G$.
\begin{enumerate}[(a)]
\item $g$ has \vocab{infinite order} means all the non-negative powers of $g$ are distinct.
\item $g$ has \vocab{order $d$} means $d$ is the smallest positive integer such that $g^d=g^0=1$.  We say $g$ has \vocab{finite order}.
\end{enumerate} 

\smallGap
We write $|g|$ or $\ord(g)$ to denote the order of $g$.
\end{dfn}

\smallGap
\begin{que}[cf. Problem 54]
What is the order of each element in $\Z_{12}$?
\end{que}
\end{frame}

% % %
\begin{frame}[c]
\begin{exe}[cf. Problem 53]\label{exe:prob53}
In $G=(\GL(2,\R),\cdot)$ find the order of each of the following elements:
\[
g=\begin{pmatrix}
	0 & -1 \\
	1 & 0 
	\end{pmatrix}
\qquad
h=\begin{pmatrix}
	3 & 5 \\
	4 & 7
	\end{pmatrix}
\qquad 
j=\begin{pmatrix}
	-\frac{1}{2} & -\frac{\sqrt 3}{2} \\
	\frac{\sqrt 3}{2} & -\frac{1}{2}
	\end{pmatrix}
\]
\end{exe}

\smallGap
\begin{exe}[cf. Problem 55]\label{exe:prob55}
Find the order of each element in the sandpile group $\mathscr S(\Gamma)$ corresponding to the graph in Figure \ref{fig:graphWithLoops}. 
\end{exe}
\end{frame}

% % % % %
\subsection[\subsecname]{Order of a group}
% % %
\begin{frame}[c]{\subsecname}
\begin{dfn}
The \vocab{order} of a group $G$ is its cardinality as a set, $|G|$.
\end{dfn}

\smallGap
\textbf{Caution:} The \emph{dihedral group $D_n$ of order $n$} has group order not $n$, but $2n$.  

\smallGap
\begin{exe}\label{exe:orderOfSn}
Find the order of the \emph{symmetric group of order $n$}, $S_n$ (it's not $n$).
\end{exe}  
\end{frame}

% % %
\begin{frame}[c]
\begin{thm}[Lagrange's Theorem]\label{thm:Lagrange}
Given a finite (order) group $G$, the order of every subgroup of $G$ divides $|G|$.
\qed
\end{thm}

\smallGap
\begin{que}
Verify Lagrange's Theorem for $\Z_{12}$. 
\end{que}
\end{frame}

% % % % %
\answerKey
% % %
\begin{frame}{\subsecname}
\exeSol[(cf. Problem 53)]{exe:prob53}
%\textbf{Solution:} 
We compute the powers of $g,h,j$:
\begin{align*}
g^2=\begin{pmatrix*}[r]
	0 & -1 \\
	1 & 0 \end{pmatrix*}
	\begin{pmatrix*}[r]
	0 & -1 \\
	1 & 0 \end{pmatrix*}
	&= \begin{pmatrix*}[r]
	-1 & 0 \\
	0 & -1 
	\end{pmatrix*}; \\
g^3=\begin{pmatrix*}[r]
	-1 & 0 \\
	0 & -1 \end{pmatrix*}
	\begin{pmatrix*}[r]
	0 & -1 \\
	1 & 0 \end{pmatrix*}
	&= \begin{pmatrix*}[r]
	0 & 1 \\
	-1 & 0 
	\end{pmatrix*}; \\
g^4 = \begin{pmatrix*}[r]
	0 & 1 \\
	-1 & 0 \end{pmatrix*}
	\begin{pmatrix*}[r]
	0 & -1 \\
	1 & 0 
	\end{pmatrix*}
	&= \begin{pmatrix*}[r]
	1 & 0 \\
	0 & 1
	\end{pmatrix*} \implies \boxed{|g|=4} \\
h^2=\begin{pmatrix}
	3 & 5 \\
	4 & 7 
	\end{pmatrix}
	\begin{pmatrix}
	3 & 5 \\
	4 & 7
	\end{pmatrix}
	&= \begin{pmatrix}
	29 & 50 \\
	40 & 69
	\end{pmatrix};
\end{align*}
Note, since the entries of $h$ are strictly positive, no power of $h$ will ever reach the identity.  Thus $\boxed{|h|=\infty}$.

\smallGap
The matrix $j$ is a rotation matrix, so we expect it to have finite degree:
\end{frame}

% % %
\begin{frame}
\begin{align*}
j^2=\begin{pmatrix}
	-\frac{1}{2} & -\frac{\sqrt 3}{2} \\
	\frac{\sqrt 3}{2} & -\frac{1}{2}
	\end{pmatrix}
	\begin{pmatrix}
	-\frac{1}{2} & -\frac{\sqrt 3}{2} \\
	\frac{\sqrt 3}{2} & -\frac{1}{2}
	\end{pmatrix}
	&=\begin{pmatrix}
	-\frac{1}{2} & \frac{\sqrt 3}{2}\\
	-\frac{\sqrt 3}{2} & -\frac{1}{2}
	\end{pmatrix}; \\
j^3 =\begin{pmatrix}
	-\frac{1}{2} & \frac{\sqrt 3}{2}\\
	-\frac{\sqrt 3}{2} & -\frac{1}{2}
	\end{pmatrix}
	\begin{pmatrix}
	-\frac{1}{2} & -\frac{\sqrt 3}{2} \\
	\frac{\sqrt 3}{2} & -\frac{1}{2}
	\end{pmatrix}
	&=\begin{pmatrix}
	1 & 0 \\
	0 & 1
	\end{pmatrix} \implies \boxed{|j|=3}
\end{align*}

\smallGap
\begin{que}
What is the rotation given by $j$?
\end{que}
\end{frame}
% % %
\begin{frame}
\exeSol[(cf. Problem 55)]{exe:prob55}
By adding various configurations to the maximal sandpile $(2,2)$ we can conclude 
\[
\mathscr S(\Gamma)=\{(1,1),(1,2),(2,1),(2,2)\}.
\]
The elements and addition table for $\mathscr S(\Gamma)$ are given in the solution to Exericse \ref{exe:prob42}.  No sandpile will have order larger than $4$ so we compute them by hand:
\begin{align*}
(1,1) \oplus (1,1) &= (2,2) \\
	\oplus \;(1,1) &= (3,3)^{\circ}=(1,2) \\
	\oplus \;(1,1) &= (2,3)^{\circ}=(2,1) \leftarrow \text{identity} \\[-0.25pc]
	\oplus \;(1,1) &= (3,2)^{\circ}=(1,1)  \implies \boxed{|(1,1)|=4}. \\
(1,2) \oplus (1,2) &= (2,4)^{\circ}=(2,2) \\
	\oplus \;(1,2) &= (3,4)^{\circ}=(1,1) \\[-0.25pc]
	\oplus \;(1,2) &= (2,3)^{\circ}=(2,1) \implies \boxed{|(1,2)|=4}. \\
(2,2) \oplus (2,2) &= (4,4)^{\circ}=(2,1) \implies \boxed{|(2,2)|=2}. 
\end{align*}
The identity, $(2,1)$, has order $1$.
\end{frame}

% % %
\begin{frame}
\exeSol{exe:orderOfSn}
%\textbf{Solution:} 
Let $S$ denote a set with $n$ distinct elements.  Each permutation in the symmetric group of order $n$ corresponds to an ordering of the members of $S$.  We count the number of orderings one position at a time.  

\smallGap
For the first position in an ordering, there are $n$ possible choices.  Then for each such choice, the second position has $n-1$ remaining choices from $S$.  Thus there are $n(n-1)$ possibilities for the first two elements in an ordering.  For each of those $n(n-1)$ possibilities the third position has $n-2$ remaining choices.  In general, having chosen the first $k<n$ elements of an ordering, the $(k+1)$st position has $n-k-1$ possible choices left in $S$.  It follows that the total number of ordering of the members of $S$ is $n(n-1)\cdots(2)(1)=\boxed{n!}$.
\end{frame}

% % % % % % % % % %
\end{document}