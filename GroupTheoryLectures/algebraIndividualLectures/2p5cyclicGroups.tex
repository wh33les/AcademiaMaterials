\documentclass[../algebraNotesMSRI-UP2016.tex]{subfiles}

\begin{document}

\section[\S \thesection]{Cyclic groups}
% % % % %
\subsection[\subsecname]{Definition}\label{subsec:cyclicDefs}
% % %
\begin{frame}[c]{\subsecname}
Recall, from Section \ref{subsec:cyclic}, the notion of a cyclic subgroup.

\smallGap
\begin{dfn}
A group $G$ is \vocab{cyclic} means there exists $g\in G$ such that $\langle g\rangle=G$.
\end{dfn}

\smallGap
\begin{ex}
Often we say ``infinite cyclic" to describe groups isomorphic to $\Z$ (see Section \ref{subsec:isomorphisms} for the definition of \vocab{isomorphism}).
\end{ex}

\smallGap
\begin{que}
What is the generator for the group $\Z$?
\end{que}
\end{frame}

% % %
\begin{frame}
\begin{ex}
$\Z\times\Z$ is not cyclic.  

\smallGap
\pf
Assume, to the contrary, that there exist $a,b\in\Z$ such that $\langle(a,b)\rangle=\Z\times\Z$.  Then for any arbitrary element $(x,y)\in\Z\times \Z$ there exists an integer $n$ such that $x=na$ and $y=nb$.  There exists another integer, $m$, such that $(x+1,y)=(ma,mb)$, which implies 
\begin{align*}
x+1 =&\ na+1 = ma \\
y =&\ nb = mb \\
\implies &\ n = m \\
\implies x+1 =&\ na+1=na,
\end{align*}
which is a contradiction.
\qed
\end{ex}
\end{frame}

% % % % %
\subsection[\subsecname]{Finite cyclic groups}
% % %
\begin{frame}[c]{\subsecname}{}
We state the \vocab{division algorithm}, as it will prove useful, particularly in Section \ref{subsec:kernelsAndImages}.

\smallGap
\begin{thm}[Division Algorithm]\label{thm:divisionAlgorithm}
Suppose $n\in\Z$ and $d\in\N$.  Then there exist unique integers $q$ (the \vocab{quotient}), and $r$ (the \vocab{remainder}) such that 
\begin{equation}\label{eq:divisionAlgorithm}
n = q\cdot d+r \quad\text{ and }\quad 0 \leq r < d.
\end{equation}
In this context, $d$ is called the \vocab{divisor}.
\qed
\end{thm}
\end{frame}

% % % 
%\begin{frame}[c]
%We shall not prove Theorem \ref{thm:divisionAlgorithm} beyond mentioning that it relies on the \vocab{Well-Ordering Principle} (which we also state without proof).  
%
%\smallGap
%\begin{thm}[Well-Ordering Principle]
%Every non-empty set of positive integers contains a smallest element. 
%\qed
%\end{thm}
%\end{frame}

% % %
\begin{frame}
%\begin{que}
%Given any arbitrary $n\in\N$, does there exist a group of order $n$?
%\end{que}
%
%\smallGap
\begin{dfn}\label{dfn:ZmodnZ}
For $n\in\N$, the set of all possible remainders upon division by $n$,
\[
\Z_n:=\{0,1,\dots,n-1\},
\]
equipped with the binary operation
\begin{align*}
\oplus_n:\Z_n\times \Z_n &\to \Z_n \\
	(a,b) &\mapsto \parbox[t]{0.5\textwidth}{
		$r$, as in Equation \eqref{eq:divisionAlgorithm}, 
		\vspace{-0.3pc}
		\flushright putting $a+b=n$ and $n=d$,}
\end{align*}
is called the \vocab{group of integers mod $n$}.
\end{dfn}
%\end{frame}
%
% % %
%\begin{frame}[c]{}
\smallGap
\begin{que}[cf. Problem 57]
For fixed $n\in\N$, which elements generate $\Z_n$ (besides 1)?
\end{que}
%
%\smallGap
%\textbf{Remark on the notation:} In many subbranches of mathematics, particularly Number Theory, the symbol $\Z_p$ is used to denote the \vocab{$p$-adic integers}, where $p$ is always a prime number.    
%
%\smallGap
%For this reason many authors prefer the notation $\Z/n\Z$, and we shall use the two interchangeably.  The alternate notation may seem unintuitive, for one, because it is a longer expression.  In Section \ref{subsec:FGAG} we reveal its origin.   
\end{frame}

% % %
\begin{frame}[c]{}{}
\begin{ex}[cf. Problem 56]
The following are ``addition tables" for each of the groups $\Z_5$ %$\Z/5\Z$ 
and $\Z_6$. %$\Z/6\Z$.
\[
\begin{array}{c | c c c c c }
 \Z_5 & 0 & 1 & 2 & 3 & 4 \\
 \hline
 0 & 0 & 1 & 2 & 3 & 4 \\ 
 1 & 1 & 2 & 3 & 4 & 0 \\
 2 & 2 & 3 & 4 & 0 & 1 \\
 3 & 3 & 4 & 0 & 1 & 2 \\
 4 & 4 & 0 & 1 & 2 & 3 
\end{array}
\hspace{2pc}
\begin{array}{c | c c c c c c }
 \Z_6  & 0 & 1 & 2 & 3 & 4 & 5 \\
 \hline
 0 & 0 & 1 & 2 & 3 & 4 & 5 \\ 
 1 & 1 & 2 & 3 & 4 & 5 & 0 \\
 2 & 2 & 3 & 4 & 5 & 0 & 1 \\
 3 & 3 & 4 & 5 & 0 & 1 & 2 \\
 4 & 4 & 5 & 0 & 1 & 2 & 3 \\
 5 & 5 & 0 & 1 & 2 & 3 & 4
\end{array}
\]	
\end{ex}
\end{frame}

% % %
\begin{frame}[c]
\begin{prop}\label{prop:subgroupOfACyclic}
Every subgroup of a cyclic group is cyclic.
\qed
\end{prop}
%
%\smallGap
%\begin{exe}\label{exe:subgroupOfACyclic}
%Prove Proposition \ref{prop:subgroupOfACyclic}.  \textit{Hint: Your proof should invoke use of both the Division Algorithm and the Well-Ordering Principle.}  
%\end{exe}

\smallGap
\begin{cor}
Every subgroup of $\Z$ is cyclic.
\qed
\end{cor}
\end{frame}

% % %
\begin{frame}
\begin{exe}[cf. Problem 58]\label{exe:prob58}
Find one generator for the subgroup
\begin{enumerate}[(a)]
\item $\langle 2,3\rangle< \Z$.
\item $\langle 4,6\rangle< \Z$.
\end{enumerate} 
\end{exe}

\smallGap
\begin{exe}[cf. Problem 59]\label{exe:prob59}
Prove every cyclic group is abelian.
\end{exe}
%\end{frame}
%
% % %
%\begin{frame}{}{}
%{\bf REMARKS ABOUT PROPOSITION \ref{prop:subgroupOfACyclic}, CYCLIC GROUPS, AND THE UPCOMING FUNDAMENTAL THEOREM OF FINITELY GENERATED ABELIAN GROUPS}
%\end{frame}
%
% % %
%\begin{frame}

\smallGap
\begin{exe}[cf. Problem 60]\label{exe:prob60}
Use Lagrange's Theorem (Theorem \ref{thm:Lagrange}) to show any group of prime order must be cyclic.
\end{exe}
\end{frame}

\begin{comment}
% % % % %
\answerKey
% % %
\begin{frame}{\subsecname}
\exeSol{exe:subgroupOfACyclic}
\end{frame}

% % %
\begin{frame}
\exeSol[(cf. Problem 58)]{exe:prob58}
\end{frame}

% % %
\begin{frame}
\exeSol[(cf. Problem 59)]{exe:prob59}
\end{frame}

% % %
\begin{frame}
\exeSol[(cf. Problem 60)]{exe:prob60}
\end{frame}

\end{comment}
% % % % % % % % % %
\end{document}