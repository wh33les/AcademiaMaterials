\documentclass[../algebraNotesMSRI-UP2016.tex]{subfiles}

\begin{document}

\section[\S \thesection]{Quotient groups}\label{sec:2p7quotientGroups}
% % % % % 
\subsection[\subsecname]{Cosets}
% % %
\begin{frame}{\subsecname}
\begin{dfn}
Let $(G,\star)$ denote a group with subgroup $H< G$ and suppose $g\in G$.  The set
\[
g\star H:=\{g\star h\mid h\in H\}
\]
is called a \vocab{(left) coset} of $H$ in $G$.  %The set
%\[
%H\star g:=\{h\star g\mid h\in H\}
%\]  
%is called a \vocab{right coset} of $H$ in $G$.
\end{dfn}

\smallGap
When using multiplicative notation we may write $gH=g\star H$; likewise with additive notation we write $g+H=g\star H$.

\smallGap
\begin{que}
What is the condition for a coset to be a subgroup?  (In general, a coset is NOT a subgroup!)
\end{que}
\end{frame}

% % %
%\begin{frame}{}{}
%{\bf EXAMPLE 18}
%\begin{ex}
%{\bf EXAMPLES WHERE LEFT AND RIGHT COSETS ARE NOT EQUAL; WHERE THEY ARE EQUAL BUT NOT ALWAYS}
%\end{ex}
%\end{frame}

%%%%%%%%%%%%%%%%%%%%%%%%%%%%%%%%%%%%%%%
\begin{comment}
% % %
\begin{frame}
\begin{dfn}\label{dfn:normal}
A subgroup $H<G$ is \vocab{normal} in $G$ means $gH=Hg$ for all $g\in G$.  We write $H\norml G$.
\end{dfn}

\smallGap 
We may also say a subgroup $H<G$ is normal if and only if $H$ is closed under \vocab{conjugacy}, meaning, $g\1Hg=H$ for all $g\in G$.  Evidently this is equivalent to Definition \ref{dfn:normal}.  The operation 
\begin{align*}
g:\{\text{subgroups in $G$}\} &\to \{\text{subgroups in $G$}\} \\
 H &\mapsto g\1Hg
\end{align*} 
for fixed $g\in G$ is called \vocab{conjugation}.  We can also conjugate individual elements $h\in G$, i.e., $g\1hg$.  %{\bf BLAH BLAH MORE AND REFS}
\end{frame}

% % %
\begin{frame}
\begin{ex}\label{ex:trivialNormal}
For any group $G$, the two \vocab{trivial} subgroups, $\langle 1_G\rangle<G$ and $G<G$, are normal.
\end{ex}

\smallGap
\begin{exe}\label{exe:normalKernel}
Prove the kernel of a group homormorphism is always normal in its source.
\end{exe}

\smallGap
\begin{exe}\label{exe:normalSubgroups}
Prove all subgroups of an abelian group are normal.
\end{exe}

\smallGap
Nearly all the groups we consider are abelian.  By default we shall work with left cosets, keeping in mind all statements we make have right coset analogues.  
\end{frame}

\end{comment}

%%%%%%%%%%%%%%%%%%%%%%%%%%%%%%%%%%%%%%%%%%%%%%%%
% % %
\begin{frame}
\begin{ex}
In $\Z_{12}$, the cosets of $H:=\{0,4,8\}=4\Z_{12}$ are:
\begin{align*}
H = 0\oplus_{12}H &= 4\oplus_{12}H = 8\oplus_{12}H = \{0,4,8\}, \\
	 1\oplus_{12}H &= 5\oplus_{12}H = 9\oplus_{12}H = \{1,5,9\},	\\
	 2\oplus_{12}H &= 6\oplus_{12}H = 10\oplus_{12}H = \{2,6,10\}, \\
	 3\oplus_{12}H &= 7\oplus_{12}H = 11\oplus_{12}H = \{3,7,11\} 
\end{align*}
\end{ex}

\smallGap
\begin{que}
What are some observations you can make?
\end{que}
\end{frame}

% % %
\begin{frame}
We use $G/H$ to denote the collection of distinct cosets of $H$ in $G$, called \vocab{$G$ modulo $H$}.  The cosets of a subgroup \vocab{partition} the group:

\smallGap
\begin{prop}\label{prop:cosetsPartition}
Let $G$ denote a group with subgroup $H< G$.
\begin{enumerate}[(a)]
\item The union of all (left, respectively, right) cosets of $H$ in $G$ is the entire group $G$.
\item For any two cosets $g\star H,h\star H\in G/H$, either
\begin{enumerate}[(i)]
	\item $g\star H=h\star H$ or
	\item $g\star H\cap h\star H=\emptyset$.
\end{enumerate}
\end{enumerate}
\qed
\end{prop}
%{\bf PROOF}
\end{frame}

% % %
\begin{frame}[c]
\begin{exe}[cf. Problem 66]\label{exe:prob66}
Let $G=\Z_{30}$ and put $H=5G$.  Using Proposition \ref{prop:cosetsPartition}, list the elements of $G/H$.  
\end{exe}

\smallGap
\begin{exe}[cf. Problem 67]\label{exe:prob67}
Let $G=\Z_2\times \Z_4$ and let $H=(1,1)G< G$.  List the elements of $H$, then list the cosets of $H$.
\end{exe}
\end{frame}

% % %
\begin{frame}
The following proposition gives a way to prove two cosets are equal.

\smallGap
\begin{prop}\label{prop:equalCosets}
Suppose $(G,\star)$ is a group with subgroup $H< G$ and suppose $g,h\in G$.  Then:
\begin{enumerate}[(a)]
\item $g\star H=H$ if and only if $g\in H$
\item $g\star H=h\star H$ if and only if $g\1h\in H$.
\item $g\star H=h\star H$ if and only if $h\in g\star H$.
\end{enumerate}
\end{prop}

\smallGap
\begin{exe}[cf. Problem 68]\label{exe:prob68}
Prove Proposition \ref{prop:equalCosets}.  \textit{Hint: Prove (a) first, then use it to prove (b), then use (b) to prove (c).}
\end{exe}
\end{frame}

% % % % %
\subsection[\subsecname]{Quotient groups}
% % %
\begin{frame}{\subsecname}
Let $G=(G,\star)$ denote an abelian group with subgroup $H< G$.  The notation used thusfar suggest a group structure on $G/H$ with a binary operation $\star_{/H}$ well-defined ``up to", or \emph{modulo} elements in $H$.  The natural choice is to define
\begin{gather}\label{eq:quotientOperation}
\begin{split}
\star_{/H}: G/H\times G/H &\to G/H \\
(g\star H,h\star H) &\mapsto (g\star h)\star H.
\end{split}
\end{gather} 

\smallGap
Given $g_1,g_2,h_1,h_2\in G$, we must verify 
\begin{align*}
(g_1\star H,h_1\star H) &=(g_2\star H,h_2\star H) \\
\implies (g_1\star H)\star_{/H}(h_1\star H) &=(g_2\star H)\star_{/H}(h_2\star H) \\
\implies (g_1\star h_1)\star H &= (g_2\star h_2)\star H.
\end{align*}
\end{frame}

% % %
\begin{frame}
Component-wise, we have, by hypothesis,% {\bf VOCAB},
\[
g_1\star H=g_2\star H\quad\text{ and }\quad h_1\star H=h_2\star H.
\]
Along with associativity,
\begin{align*}
(g_1\star h_1)\star H = g_1\star(h_1\star H) \\
	= g_1\star(h_2\star H) &= g_1\star(H\star h_2) %\tag{\alert{!!!}}
	\\
	&= (g_1\star H)\star h_2 \\
	&= (g_2\star H)\star h_2 \\
	&= g_2\star (H\star h_2) = g_2\star (h_2\star H) %\tag{\alert{!!!}}
	\\
	&\phantom{= g_2\star (H\star h_2)} \;=(g_2\star h_2)\star H.
\end{align*}
%provided the lines tagged with (\alert{!!!}) are correct.
%\begin{que}
%What additional hypothesis do we need?
%\end{que}
\end{frame}

% % %
\begin{frame}[c]
\begin{thm}\label{thm:quotientGroup}
Let $G=(G,\star)$ denote an abelian group with subgroup $H< G$.  The set $G/H$ is a group, called the \vocab{quotient group} of $G$ by $H$, equipped with the operation $\star_{/H}$ definined in Equation \eqref{eq:quotientOperation}.
\qed
\end{thm}

%\smallGap
%\begin{exe}\label{exe:quotientGroup}
%Prove Theorem \ref{thm:quotientGroup}.
%\end{exe}
%
%\smallGap
%The alternate notation $\Z/n\Z$ to $\Z_n$ arises exactly because it is a quotient group.  A more rigorous statement is given in Section \ref{sec:2p8firstIsomorphismTheorem}. 
\end{frame}

% % % 
\begin{frame}[c]
\begin{exe}[cf. Problems 69-70]\label{exe:probs69-70}
Write down the addition table for $G/H$ in 
\begin{enumerate}[(a)]
\item Exercise \ref{exe:prob66}.
\item Exercise \ref{exe:prob67}.
\end{enumerate}
\end{exe}

%{\bf EXAMPLE 21}
\end{frame}

% % % % %
\subsection[\subsecname]{Non-obvious isomorphisms}
% % %
\begin{frame}{\subsecname}
\begin{ex}\label{ex:isoToZ6}
Let $G=\Z\times \Z$ and define 
\begin{align*}
H=(3,0)G+(0,2)G &= \{m(3,0)+n(0,2)\mid m,n\in \Z\} \\
	&= \{(3m,2n)\mid m,n\in \Z\}.
\end{align*}
\end{ex}

\smallGap
Think of the elements in $H$ as movements on a grid indexed by $\Z\times \Z$.  The generator $(3,0)$ is right by 3; the generator $(0,2)$ is up by 2.  

\smallGap 
The cosets of $H$ in $G$ are: 
\smallGap
\[
\begin{aligned}
\die{6} &:=(0,0)+H=H \\
\die{1} &:=(1,1)+H \\
\die{2} &:=(2,0)+H
\end{aligned} \qquad	
\begin{aligned}
\die{3} &:=(0,1)+H \\
\die{4} &:=(1,0)+H \\
\die{5} &:=(2,1)+H
\end{aligned}
\]
\end{frame}

% % %
\begin{frame}
Compare the addition table for $G/H$ to the one for $\Z/6\Z$:
\[
\begin{array}{c | c c c c c c }
 G/H  & \die{6} & \die{1} & \die{2} & \die{3} & \die{4} & \die{5} \\
 \hline
 \die{6} & \die{6} & \die{1} & \die{2} & \die{3} & \die{4} & \die{5} \\ 
 \die{1} & \die{1} & \die{2} & \die{3} & \die{4} & \die{5} & \die{6} \\
 \die{2} & \die{2} & \die{3} & \die{4} & \die{5} & \die{6} & \die{1} \\
 \die{3} & \die{3} & \die{4} & \die{5} & \die{6} & \die{1} & \die{2} \\
 \die{4} & \die{4} & \die{5} & \die{6} & \die{1} & \die{2} & \die{3} \\
 \die{5} & \die{5} & \die{6} & \die{1} & \die{2} & \die{3} & \die{4}
\end{array}
\qquad
\begin{array}{c | c c c c c c }
 \Z/6\Z  & 0 & 1 & 2 & 3 & 4 & 5 \\
 \hline
 0 & 0 & 1 & 2 & 3 & 4 & 5 \\ 
 1 & 1 & 2 & 3 & 4 & 5 & 0 \\
 2 & 2 & 3 & 4 & 5 & 0 & 1 \\
 3 & 3 & 4 & 5 & 0 & 1 & 2 \\
 4 & 4 & 5 & 0 & 1 & 2 & 3 \\
 5 & 5 & 0 & 1 & 2 & 3 & 4
\end{array}
\]	

\smallGap
\textbf{Conclusion:} $G/H\cong \Z/6\Z$ via the correspondence in the addition tables.

\smallGap
\begin{que}
What else is $G/H$ isomorphic to?
\end{que}
\end{frame}

% % %
%\begin{frame}
%\begin{ex}
%{\bf EXAMPLE $\Z^2/\langle(2,-2),(2,2)\rangle$.}
%\end{ex}
%\end{frame}

% % %
\begin{frame}[c]
\begin{exe}[cf. Problem 72]\label{exe:prob72}
``Simplify" the following \vocab{group presentations} (a term we define in Section \ref{sec:2p10FTOFinitelyGeneratedAbelianGroups}) by exhibting an isomorphism in each case.
\begin{enumerate}
\item $\Z\times \Z/\langle(1,1)\rangle$
\item\label{exept:prob72-2} $\Z\times \Z/\langle(2,-1),(-1,2)\rangle$
\end{enumerate} 
\end{exe}
\end{frame}

% % % % %
%\subsection[\subsecname]{Quotient means kill}
% % %
%\begin{frame}[c]{\subsecname}
%We saw in Example \ref{ex:ZmodnZ}, the (group) homomorphism
%\begin{align*}
%\varphi : \Z &\to \Z/n\Z \\
%m &\mapsto m\mod n
%\end{align*}
%is surjective, and $\ker{\varphi}=n\Z$.  We say $\varphi$ \vocab{kills} $n$.  
%
%\smallGap
%Generalizing this notion:
%\begin{exe}[cf. Problem 73]\label{exe:prob73}
%Prove that if $H\norml G$ then the natural group homomorphism 
%\begin{align*}
%\bar{\psi}: G &\to G/H \\
%	g &\mapsto gH
%\end{align*}
%is surjective.  Colloquially, when we apply $\bar{\psi}$ we say we are \emph{killing} the subgroup $H$.  %{\bf ADDRESS THE BAR}
%\end{exe}
%\end{frame}

\begin{comment}
% % % % %
\answerKey
% % %
\begin{frame}{\subsecname}
\exeSol{exe:normalKernel}
\end{frame}

% % %
\begin{frame}
\exeSol{exe:normalSubgroups}
\end{frame}

% % %
\begin{frame}
\exeSol[(cf. Problem 66)]{exe:prob66}
\end{frame}

% % %
\begin{frame}
\exeSol[(cf. Problem 67)]{exe:prob67}
\end{frame}

% % %
\begin{frame}
\exeSol[(cf. Problem 68)]{exe:prob68}
\end{frame}

% % %
\begin{frame}
\exeSol{exe:quotientGroup}
\end{frame}

% % %
\begin{frame}
\exeSol[(cf. Problems 69-70)]{exe:probs69-70}
\end{frame}

% % %
\begin{frame}
\exeSol[(cf. Problem 72)]{exe:prob72}
\end{frame}

% % %
\begin{frame}
\exeSol[(cf. Problem 73)]{exe:prob73}
\end{frame}

\end{comment}
% % % % % % % % % %
\end{document}